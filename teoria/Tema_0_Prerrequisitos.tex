% Options for packages loaded elsewhere
\PassOptionsToPackage{unicode}{hyperref}
\PassOptionsToPackage{hyphens}{url}
\PassOptionsToPackage{dvipsnames,svgnames,x11names}{xcolor}
%
\documentclass[
  ignorenonframetext,
  aspectratio=169]{beamer}
\usepackage{pgfpages}
\setbeamertemplate{caption}[numbered]
\setbeamertemplate{caption label separator}{: }
\setbeamercolor{caption name}{fg=normal text.fg}
\beamertemplatenavigationsymbolsempty
% Prevent slide breaks in the middle of a paragraph
\widowpenalties 1 10000
\raggedbottom
\setbeamertemplate{part page}{
  \centering
  \begin{beamercolorbox}[sep=16pt,center]{part title}
    \usebeamerfont{part title}\insertpart\par
  \end{beamercolorbox}
}
\setbeamertemplate{section page}{
  \centering
  \begin{beamercolorbox}[sep=12pt,center]{part title}
    \usebeamerfont{section title}\insertsection\par
  \end{beamercolorbox}
}
\setbeamertemplate{subsection page}{
  \centering
  \begin{beamercolorbox}[sep=8pt,center]{part title}
    \usebeamerfont{subsection title}\insertsubsection\par
  \end{beamercolorbox}
}
\AtBeginPart{
  \frame{\partpage}
}
\AtBeginSection{
  \ifbibliography
  \else
    \frame{\sectionpage}
  \fi
}
\AtBeginSubsection{
  \frame{\subsectionpage}
}

\usepackage{amsmath,amssymb}
\usepackage{iftex}
\ifPDFTeX
  \usepackage[T1]{fontenc}
  \usepackage[utf8]{inputenc}
  \usepackage{textcomp} % provide euro and other symbols
\else % if luatex or xetex
  \usepackage{unicode-math}
  \defaultfontfeatures{Scale=MatchLowercase}
  \defaultfontfeatures[\rmfamily]{Ligatures=TeX,Scale=1}
\fi
\usepackage{lmodern}
\usetheme[]{CambridgeUS}
\ifPDFTeX\else  
    % xetex/luatex font selection
\fi
% Use upquote if available, for straight quotes in verbatim environments
\IfFileExists{upquote.sty}{\usepackage{upquote}}{}
\IfFileExists{microtype.sty}{% use microtype if available
  \usepackage[]{microtype}
  \UseMicrotypeSet[protrusion]{basicmath} % disable protrusion for tt fonts
}{}
\makeatletter
\@ifundefined{KOMAClassName}{% if non-KOMA class
  \IfFileExists{parskip.sty}{%
    \usepackage{parskip}
  }{% else
    \setlength{\parindent}{0pt}
    \setlength{\parskip}{6pt plus 2pt minus 1pt}}
}{% if KOMA class
  \KOMAoptions{parskip=half}}
\makeatother
\usepackage{xcolor}
\newif\ifbibliography
\setlength{\emergencystretch}{3em} % prevent overfull lines
\setcounter{secnumdepth}{-\maxdimen} % remove section numbering

\usepackage{color}
\usepackage{fancyvrb}
\newcommand{\VerbBar}{|}
\newcommand{\VERB}{\Verb[commandchars=\\\{\}]}
\DefineVerbatimEnvironment{Highlighting}{Verbatim}{commandchars=\\\{\}}
% Add ',fontsize=\small' for more characters per line
\usepackage{framed}
\definecolor{shadecolor}{RGB}{241,243,245}
\newenvironment{Shaded}{\begin{snugshade}}{\end{snugshade}}
\newcommand{\AlertTok}[1]{\textcolor[rgb]{0.68,0.00,0.00}{#1}}
\newcommand{\AnnotationTok}[1]{\textcolor[rgb]{0.37,0.37,0.37}{#1}}
\newcommand{\AttributeTok}[1]{\textcolor[rgb]{0.40,0.45,0.13}{#1}}
\newcommand{\BaseNTok}[1]{\textcolor[rgb]{0.68,0.00,0.00}{#1}}
\newcommand{\BuiltInTok}[1]{\textcolor[rgb]{0.00,0.23,0.31}{#1}}
\newcommand{\CharTok}[1]{\textcolor[rgb]{0.13,0.47,0.30}{#1}}
\newcommand{\CommentTok}[1]{\textcolor[rgb]{0.37,0.37,0.37}{#1}}
\newcommand{\CommentVarTok}[1]{\textcolor[rgb]{0.37,0.37,0.37}{\textit{#1}}}
\newcommand{\ConstantTok}[1]{\textcolor[rgb]{0.56,0.35,0.01}{#1}}
\newcommand{\ControlFlowTok}[1]{\textcolor[rgb]{0.00,0.23,0.31}{#1}}
\newcommand{\DataTypeTok}[1]{\textcolor[rgb]{0.68,0.00,0.00}{#1}}
\newcommand{\DecValTok}[1]{\textcolor[rgb]{0.68,0.00,0.00}{#1}}
\newcommand{\DocumentationTok}[1]{\textcolor[rgb]{0.37,0.37,0.37}{\textit{#1}}}
\newcommand{\ErrorTok}[1]{\textcolor[rgb]{0.68,0.00,0.00}{#1}}
\newcommand{\ExtensionTok}[1]{\textcolor[rgb]{0.00,0.23,0.31}{#1}}
\newcommand{\FloatTok}[1]{\textcolor[rgb]{0.68,0.00,0.00}{#1}}
\newcommand{\FunctionTok}[1]{\textcolor[rgb]{0.28,0.35,0.67}{#1}}
\newcommand{\ImportTok}[1]{\textcolor[rgb]{0.00,0.46,0.62}{#1}}
\newcommand{\InformationTok}[1]{\textcolor[rgb]{0.37,0.37,0.37}{#1}}
\newcommand{\KeywordTok}[1]{\textcolor[rgb]{0.00,0.23,0.31}{#1}}
\newcommand{\NormalTok}[1]{\textcolor[rgb]{0.00,0.23,0.31}{#1}}
\newcommand{\OperatorTok}[1]{\textcolor[rgb]{0.37,0.37,0.37}{#1}}
\newcommand{\OtherTok}[1]{\textcolor[rgb]{0.00,0.23,0.31}{#1}}
\newcommand{\PreprocessorTok}[1]{\textcolor[rgb]{0.68,0.00,0.00}{#1}}
\newcommand{\RegionMarkerTok}[1]{\textcolor[rgb]{0.00,0.23,0.31}{#1}}
\newcommand{\SpecialCharTok}[1]{\textcolor[rgb]{0.37,0.37,0.37}{#1}}
\newcommand{\SpecialStringTok}[1]{\textcolor[rgb]{0.13,0.47,0.30}{#1}}
\newcommand{\StringTok}[1]{\textcolor[rgb]{0.13,0.47,0.30}{#1}}
\newcommand{\VariableTok}[1]{\textcolor[rgb]{0.07,0.07,0.07}{#1}}
\newcommand{\VerbatimStringTok}[1]{\textcolor[rgb]{0.13,0.47,0.30}{#1}}
\newcommand{\WarningTok}[1]{\textcolor[rgb]{0.37,0.37,0.37}{\textit{#1}}}

\providecommand{\tightlist}{%
  \setlength{\itemsep}{0pt}\setlength{\parskip}{0pt}}\usepackage{longtable,booktabs,array}
\usepackage{calc} % for calculating minipage widths
\usepackage{caption}
% Make caption package work with longtable
\makeatletter
\def\fnum@table{\tablename~\thetable}
\makeatother
\usepackage{graphicx}
\makeatletter
\def\maxwidth{\ifdim\Gin@nat@width>\linewidth\linewidth\else\Gin@nat@width\fi}
\def\maxheight{\ifdim\Gin@nat@height>\textheight\textheight\else\Gin@nat@height\fi}
\makeatother
% Scale images if necessary, so that they will not overflow the page
% margins by default, and it is still possible to overwrite the defaults
% using explicit options in \includegraphics[width, height, ...]{}
\setkeys{Gin}{width=\maxwidth,height=\maxheight,keepaspectratio}
% Set default figure placement to htbp
\makeatletter
\def\fps@figure{htbp}
\makeatother

\usepackage{colortbl}
\usepackage{amsmath,color,array,booktabs,algorithm2e}
 \usepackage{mathdots}
 %\usepackage{yhmath}
 %\usepackage{MnSymbol}



 \definecolor{LightBlue}{RGB}{173,216,230}
 \newcommand\blue[1]{\textcolor{blue}{#1}}
 \newcommand\red[1]{\textcolor{red}{#1}}
 \newcommand\green[1]{\textcolor{green}{#1}}

\renewcommand{\contentsname}{Índice}
\renewcommand{\chaptername}{Parte}
\renewcommand{\sectionname}{Lección}
\AtBeginDocument{\title[Tema 0: Prerrequisitos...]{Tema 0: Prerrequisitos. Teoría de conjuntos y combinatoria}}
\makeatletter
\makeatother
\makeatletter
\makeatother
\makeatletter
\@ifpackageloaded{caption}{}{\usepackage{caption}}
\AtBeginDocument{%
\ifdefined\contentsname
  \renewcommand*\contentsname{Tabla de contenidos}
\else
  \newcommand\contentsname{Tabla de contenidos}
\fi
\ifdefined\listfigurename
  \renewcommand*\listfigurename{Listado de Figuras}
\else
  \newcommand\listfigurename{Listado de Figuras}
\fi
\ifdefined\listtablename
  \renewcommand*\listtablename{Listado de Tablas}
\else
  \newcommand\listtablename{Listado de Tablas}
\fi
\ifdefined\figurename
  \renewcommand*\figurename{Figura}
\else
  \newcommand\figurename{Figura}
\fi
\ifdefined\tablename
  \renewcommand*\tablename{Tabla}
\else
  \newcommand\tablename{Tabla}
\fi
}
\@ifpackageloaded{float}{}{\usepackage{float}}
\floatstyle{ruled}
\@ifundefined{c@chapter}{\newfloat{codelisting}{h}{lop}}{\newfloat{codelisting}{h}{lop}[chapter]}
\floatname{codelisting}{Listado}
\newcommand*\listoflistings{\listof{codelisting}{Listado de Listados}}
\makeatother
\makeatletter
\@ifpackageloaded{caption}{}{\usepackage{caption}}
\@ifpackageloaded{subcaption}{}{\usepackage{subcaption}}
\makeatother
\makeatletter
\@ifpackageloaded{tcolorbox}{}{\usepackage[skins,breakable]{tcolorbox}}
\makeatother
\makeatletter
\@ifundefined{shadecolor}{\definecolor{shadecolor}{rgb}{.97, .97, .97}}
\makeatother
\makeatletter
\makeatother
\makeatletter
\ifdefined\Shaded\renewenvironment{Shaded}{\begin{tcolorbox}[borderline west={3pt}{0pt}{shadecolor}, interior hidden, enhanced, frame hidden, sharp corners, boxrule=0pt, breakable]}{\end{tcolorbox}}\fi
\makeatother
\makeatletter
\makeatother
\ifLuaTeX
\usepackage[bidi=basic]{babel}
\else
\usepackage[bidi=default]{babel}
\fi
\babelprovide[main,import]{spanish}
% get rid of language-specific shorthands (see #6817):
\let\LanguageShortHands\languageshorthands
\def\languageshorthands#1{}
\ifLuaTeX
  \usepackage{selnolig}  % disable illegal ligatures
\fi
\IfFileExists{bookmark.sty}{\usepackage{bookmark}}{\usepackage{hyperref}}
\IfFileExists{xurl.sty}{\usepackage{xurl}}{} % add URL line breaks if available
\urlstyle{same} % disable monospaced font for URLs
\hypersetup{
  pdftitle={Tema 0: Prerrequisitos. Teoría de conjuntos y combinatoria},
  pdfauthor={Parte 1: Probabilidad con R y python},
  pdflang={es},
  colorlinks=true,
  linkcolor={blue},
  filecolor={Maroon},
  citecolor={Blue},
  urlcolor={Blue},
  pdfcreator={LaTeX via pandoc}}

\title{Tema 0: Prerrequisitos. Teoría de conjuntos y combinatoria}
\author{Parte 1: Probabilidad con R y python}
\date{agosto 2023}

\begin{document}
\frame{\titlepage}
\hypertarget{antes-de-empezar}{%
\section{Antes de empezar}\label{antes-de-empezar}}

\begin{frame}{Consideraciones}
\protect\hypertarget{consideraciones}{}
Para aprender cálculo de probabilidades son necesarios conocimientos de:

\begin{enumerate}
\tightlist
\item
  Cálculo: Derivadas, integrales, límites, sumas de series\ldots{}
\item
  Geometría básica y álgebra lineal : rectas, hiperplanos,
  volúmenes\ldots{} Matrices, valores propios\ldots{}
\item
  Teoría de conjuntos y combinatoria\ldots..
\end{enumerate}
\end{frame}

\begin{frame}{Consideraciones}
\protect\hypertarget{consideraciones-1}{}
Por experiencia sabemos que la mayoría de estudiantes tienen más
conocimientos de cálculo, geometría y matrices.

Pero muchos tienen una falta de conocimientos en teoría básica de
conjuntos y combinatoria (matemática discreta).
\end{frame}

\begin{frame}{Teoría de conjuntos}
\protect\hypertarget{teoruxeda-de-conjuntos}{}
\textbf{Definición de conjunto}

La definición de conjunto es una
\href{https://es.wikipedia.org/wiki/Concepto_primitivo}{idea o noción
primitiva}. Es decir es una idea básica del pensamiento humano: un
conjunto es una colección de objetos: números, imágenes\ldots{}
cualquier cosa, jugadores de fútbol, palabras, colores \ldots.

La teoría de conjuntos básicas es simple y natural y es la que
necesitamos para este curso.

La teoría de conjuntos matemática es más compleja y presenta varias
paradojas como la
\href{https://es.wikipedia.org/wiki/Paradoja_de_Russell}{paradoja de
Russell}.
\end{frame}

\begin{frame}{Teoría de conjuntos}
\protect\hypertarget{teoruxeda-de-conjuntos-1}{}
La idea o noción práctica de conjunto es la de una colección de objetos
de un cierto tipo.

Estas colecciones o conjuntos se pueden definir por:

\begin{itemize}
\tightlist
\item
  \green{Comprensión}: reuniendo los objetos que cumplen una propiedad
  \(p\)
\item
  \green{Extensión}: dando una lista exhaustiva de los miembros del
  conjunto
\end{itemize}
\end{frame}

\begin{frame}{Conjuntos básicos}
\protect\hypertarget{conjuntos-buxe1sicos}{}
Los conjuntos suelen tener un conjunto madre como por ejemplo

\begin{itemize}
\tightlist
\item
  \(\mathbb{N}=\{0,1,2,\ldots\}\)
\item
  \(\mathbb{Z}=\{\ldots,-2,-1,0,1,2,\ldots\}\)
\item
  \(\mathbb{Q}=\left\{\frac{p}{q}\quad\Big|\quad p,q\in \mathbb{Z} \mbox{ y } q \not= 0.\right\}\)
\item
  \(\mathbb{R}=\{\mbox{Todos los puntos de una recta.}\}\)
\end{itemize}
\end{frame}

\begin{frame}{Conjuntos básicos}
\protect\hypertarget{conjuntos-buxe1sicos-1}{}
\begin{itemize}
\tightlist
\item
  \(\mathbb{C}= \left\{a+b\cdot i\quad \big|\quad a,b\in \mathbb{R}\right\}\mbox{ los números complejos}\quad a+b\cdot i.\)
\item
  Alfabeto = \(\{a,b,c,\ldots, A,B,C,\ldots\}.\)
\item
  Palabras = \(\{paz, guerra, amor, probabilidad,\ldots\}.\)
\end{itemize}

Recordemos que \(i\) es la unidad imaginaria que cumple que
\(i=\sqrt{-1}\).
\end{frame}

\begin{frame}{Características y propiedades básicas de los conjuntos}
\protect\hypertarget{caracteruxedsticas-y-propiedades-buxe1sicas-de-los-conjuntos}{}
\begin{itemize}
\tightlist
\item
  Si a cada objeto \(x\) de \(\Omega\) le llamaremos \textbf{elemento
  del conjunto} \(\Omega\) y diremos que \(x\) pertenece a \(\Omega\).
  Lo denotaremos por \(x\in \Omega\).
\item
  Un \textbf{conjunto de un elemento}, por ejemplo \(\{1\}\) recibe el
  nombre de \textbf{conjunto elemental} (o \textbf{singleton} del
  inglés).
\item
  Sea \(A\) otro conjunto diremos que \(A\) \textbf{es igual a} \(B\) si
  todos los elementos \(A\) están en \(B\) y todos los elementos de
  \(B\) están en \(A\). Por ejemplo \(A=\{1,2,3\}\) es igual a
  \(B=\{3,1,2\}\).
\end{itemize}
\end{frame}

\begin{frame}{Características y propiedades básicas de los conjuntos}
\protect\hypertarget{caracteruxedsticas-y-propiedades-buxe1sicas-de-los-conjuntos-1}{}
\begin{itemize}
\tightlist
\item
  Si \(B\) es otro conjunto, tal que si \(x\in A\) entonces \(x\in B\)
  diremos que \(A\) es un subconjunto de o que está contenido en \(B\).
  Lo denotaremos por \(A\subseteq B.\)
\item
  El conjunto que no tiene elementos se denomina conjunto vacío y se
  denota por el símbolo \(\emptyset\).
\item
  Dado \(A\) un conjunto cualquiera obviamente \(\emptyset\subseteq A.\)
\end{itemize}
\end{frame}

\begin{frame}{Características y propiedades básicas de los conjuntos}
\protect\hypertarget{caracteruxedsticas-y-propiedades-buxe1sicas-de-los-conjuntos-2}{}
Tomemos como conjunto base \(\Omega=\{1,2,3\}\)

\begin{itemize}
\tightlist
\item
  \(\Omega\) es un conjunto de cardinal 3, se denota por
  \(\#(\Omega)=3\) o por \(|\Omega|=3\)
\item
  El conjunto \(\Omega\) tiene \(2^3=8\) subconjuntos.

  \begin{itemize}
  \tightlist
  \item
    el vacio \(\emptyset\) y los elementales \(\{1\},\{2\},\{3\}\)
  \item
    los subconjuntos de dos elementos: \(\{1,2\},\{1,3\},\{2,3\}\)
  \item
    el conjunto total de tres elementos \(\Omega=\{1,2,3\}.\)
  \end{itemize}
\end{itemize}
\end{frame}

\begin{frame}{Características y propiedades básicas de los conjuntos}
\protect\hypertarget{caracteruxedsticas-y-propiedades-buxe1sicas-de-los-conjuntos-3}{}
Dado un conjunto \(\Omega\) podemos construir el \textbf{conjunto de
todas sus partes} (todos sus subconjuntos) al que denotamos por
\(\mathcal{P}(\Omega)\). También se denomina de forma directa partes de
\(\Omega\).

Cardinal de las partes de un conjunto

El \textbf{cardinal de la partes de un conjunto} es
\(\#(\mathcal{P}(\Omega))=2^{\#(\Omega)}.\)
\end{frame}

\begin{frame}{Características y propiedades básicas de los conjuntos}
\protect\hypertarget{caracteruxedsticas-y-propiedades-buxe1sicas-de-los-conjuntos-4}{}
Por ejemplo
\(\#\left(\mathcal{P}(\{1,2,3\})\right)=2^{\#(\{1,2,3\})}=2^3=8.\)

Efectivamente

\[\mathcal{P}(\{1,2,3\})=\{\emptyset,\{1\},\{2\},\{3\},\{1,2\},\{1,3\},\{2,3\},\{1,2,3\}\}.\]
\end{frame}

\begin{frame}{Características y propiedades básicas de los conjuntos}
\protect\hypertarget{caracteruxedsticas-y-propiedades-buxe1sicas-de-los-conjuntos-5}{}
Dado un subconjunto \(A\) de \(\Omega\) podemos construir la función
característica de \(A\) \[\chi_A:\Omega \to \{0,1\}\]

dado un \(\omega\in \Omega\)

\[
\chi_A(\omega)=
\left\{
\begin{array}{ll}
1 &  \mbox{si }\omega \in A\\
0 &  \mbox{si }\omega \not\in A
\end{array}
\right.
\]
\end{frame}

\begin{frame}{Operaciones conjuntos: Intersección.}
\protect\hypertarget{operaciones-conjuntos-intersecciuxf3n.}{}
Sea \(\Omega\) un conjunto y \(A\) y \(B\) dos subconjuntos de
\(\Omega\).

El conjunto \textbf{intersección} de \(A\) y \(B\) es el formado por
todos los elementos que perteneces a \(A\) \textbf{Y} \(B\), se denota
por \(A\cap B\).

Más formalmente

\[
A\cap B=\left\{x\in\Omega \big| x\in A \mbox{ y } x\in B\right\}.
\]
\end{frame}

\begin{frame}{Operaciones conjuntos: Unión.}
\protect\hypertarget{operaciones-conjuntos-uniuxf3n.}{}
El conjunto \textbf{unión} de \(A\) y \(B\) es el formado por todos los
elementos que perteneces a \(A\) \textbf{O} pertenecen a \(B\), se
denota por \(A\cup B\).

Más formalmente

\[
A\cup B=\left\{x\in\Omega \big| x\in A \mbox{ o } x\in B\right\}.
\]
\end{frame}

\begin{frame}{Operaciones conjuntos: Diferencia.}
\protect\hypertarget{operaciones-conjuntos-diferencia.}{}
El conjunto \textbf{diferencia} de \(A\) y \(B\) es el formado por todos
los elementos que perteneces a \(A\) \textbf{Y NO} pertenecen a \(B\),
se denota por \(A-B=A-(A\cap B)\).

Más formalmente

\[
A- B=\left\{x\in\Omega \big| x\in A \mbox{ y } x\notin B\right\}.
\]
\end{frame}

\begin{frame}{Operaciones conjuntos: Complementario}
\protect\hypertarget{operaciones-conjuntos-complementario}{}
El \textbf{complementario} de un subconjunto \(A\) de \(\Omega\) es
\(\Omega-A\) y se denota por \(A^c\) o \(\overline{A}\).

Más formalmente

\[
A^c=\left\{x\in\Omega \big| x\not\in A\right\}.
\]
\end{frame}

\begin{frame}{Más propiedades y definiciones}
\protect\hypertarget{muxe1s-propiedades-y-definiciones}{}
Sea \(\Omega\) un conjunto y \(A\), \(B\), \(C\) tres subconjuntos de
\(\Omega\)

\begin{itemize}
\tightlist
\item
  Se dice que dos conjuntos \(A\) y \(B\) \textbf{son disjuntos} si
  \(A\cap B=\emptyset.\)
\item
  \(\Omega^c=\emptyset\).
\item
  \(\emptyset^c=\Omega\).
\item
  \(A\cup B=B \cup A\) , \(A\cap B=B\cap A\) conmutativas.
\item
  \((A\cup B) \cup C = A \cup( B \cup C)\) ,
  \((A\cap B) \cap C = A \cap( B \cap C)\) asociativas.
\item
  \(A\cup (B\cap C)=(A\cup B) \cap (A\cup C)\) ,
  \(A\cap (B\cup C)=(A\cap B) \cup (A\cap C)\) distributivas.
\item
  \(\left(A^c\right)^c=A\) doble complementario.
\item
  \(\left(A\cup B\right)^c=A^c \cap B^c\),
  \(\left(A\cap B\right)^c=A^c \cup B^c\)
  \href{https://es.wikipedia.org/wiki/Leyes_de_De_Morgan}{leyes de De
  Morgan}.
\end{itemize}
\end{frame}

\begin{frame}[fragile]{Con R, ejemplos.}
\protect\hypertarget{con-r-ejemplos.}{}
Con R los conjuntos de pueden definir como vectores

\begin{Shaded}
\begin{Highlighting}[]
\NormalTok{Omega}\OtherTok{=}\FunctionTok{c}\NormalTok{(}\DecValTok{1}\NormalTok{,}\DecValTok{2}\NormalTok{,}\DecValTok{3}\NormalTok{,}\DecValTok{4}\NormalTok{,}\DecValTok{5}\NormalTok{,}\DecValTok{6}\NormalTok{,}\DecValTok{7}\NormalTok{,}\DecValTok{8}\NormalTok{,}\DecValTok{9}\NormalTok{,}\DecValTok{10}\NormalTok{)}
\NormalTok{A}\OtherTok{=}\FunctionTok{c}\NormalTok{(}\DecValTok{1}\NormalTok{,}\DecValTok{2}\NormalTok{,}\DecValTok{3}\NormalTok{,}\DecValTok{4}\NormalTok{,}\DecValTok{5}\NormalTok{)}
\NormalTok{B}\OtherTok{=}\FunctionTok{c}\NormalTok{(}\DecValTok{1}\NormalTok{,}\DecValTok{4}\NormalTok{,}\DecValTok{5}\NormalTok{)}
\NormalTok{C}\OtherTok{=}\FunctionTok{c}\NormalTok{(}\DecValTok{4}\NormalTok{,}\DecValTok{6}\NormalTok{,}\DecValTok{7}\NormalTok{,}\DecValTok{8}\NormalTok{)}
\NormalTok{Omega}
\end{Highlighting}
\end{Shaded}

\begin{verbatim}
 [1]  1  2  3  4  5  6  7  8  9 10
\end{verbatim}
\end{frame}

\begin{frame}[fragile]{Con R, ejemplos.}
\protect\hypertarget{con-r-ejemplos.-1}{}
\begin{Shaded}
\begin{Highlighting}[]
\NormalTok{A}
\end{Highlighting}
\end{Shaded}

\begin{verbatim}
[1] 1 2 3 4 5
\end{verbatim}

\begin{Shaded}
\begin{Highlighting}[]
\NormalTok{B}
\end{Highlighting}
\end{Shaded}

\begin{verbatim}
[1] 1 4 5
\end{verbatim}

\begin{Shaded}
\begin{Highlighting}[]
\NormalTok{C}
\end{Highlighting}
\end{Shaded}

\begin{verbatim}
[1] 4 6 7 8
\end{verbatim}
\end{frame}

\begin{frame}[fragile]{Con R, ejemplos.}
\protect\hypertarget{con-r-ejemplos.-2}{}
\(A\cap B\)

\begin{Shaded}
\begin{Highlighting}[]
\NormalTok{A}
\end{Highlighting}
\end{Shaded}

\begin{verbatim}
[1] 1 2 3 4 5
\end{verbatim}

\begin{Shaded}
\begin{Highlighting}[]
\NormalTok{B}
\end{Highlighting}
\end{Shaded}

\begin{verbatim}
[1] 1 4 5
\end{verbatim}

\begin{Shaded}
\begin{Highlighting}[]
\FunctionTok{intersect}\NormalTok{(A,B)}
\end{Highlighting}
\end{Shaded}

\begin{verbatim}
[1] 1 4 5
\end{verbatim}
\end{frame}

\begin{frame}[fragile]{Con R, ejemplos.}
\protect\hypertarget{con-r-ejemplos.-3}{}
\(A\cup B\)

\begin{Shaded}
\begin{Highlighting}[]
\NormalTok{A}
\end{Highlighting}
\end{Shaded}

\begin{verbatim}
[1] 1 2 3 4 5
\end{verbatim}

\begin{Shaded}
\begin{Highlighting}[]
\NormalTok{B}
\end{Highlighting}
\end{Shaded}

\begin{verbatim}
[1] 1 4 5
\end{verbatim}

\begin{Shaded}
\begin{Highlighting}[]
\FunctionTok{union}\NormalTok{(A,B)}
\end{Highlighting}
\end{Shaded}

\begin{verbatim}
[1] 1 2 3 4 5
\end{verbatim}
\end{frame}

\begin{frame}[fragile]{Con R, ejemplos.}
\protect\hypertarget{con-r-ejemplos.-4}{}
\(B-C\)

\begin{Shaded}
\begin{Highlighting}[]
\NormalTok{B}
\end{Highlighting}
\end{Shaded}

\begin{verbatim}
[1] 1 4 5
\end{verbatim}

\begin{Shaded}
\begin{Highlighting}[]
\NormalTok{C}
\end{Highlighting}
\end{Shaded}

\begin{verbatim}
[1] 4 6 7 8
\end{verbatim}

\begin{Shaded}
\begin{Highlighting}[]
\FunctionTok{setdiff}\NormalTok{(B,C)}
\end{Highlighting}
\end{Shaded}

\begin{verbatim}
[1] 1 5
\end{verbatim}
\end{frame}

\begin{frame}[fragile]{Con R, ejemplos.}
\protect\hypertarget{con-r-ejemplos.-5}{}
\(A^c=\Omega-A\)

\begin{Shaded}
\begin{Highlighting}[]
\NormalTok{Omega}
\end{Highlighting}
\end{Shaded}

\begin{verbatim}
 [1]  1  2  3  4  5  6  7  8  9 10
\end{verbatim}

\begin{Shaded}
\begin{Highlighting}[]
\NormalTok{A}
\end{Highlighting}
\end{Shaded}

\begin{verbatim}
[1] 1 2 3 4 5
\end{verbatim}

\begin{Shaded}
\begin{Highlighting}[]
\FunctionTok{setdiff}\NormalTok{(Omega,A)}
\end{Highlighting}
\end{Shaded}

\begin{verbatim}
[1]  6  7  8  9 10
\end{verbatim}
\end{frame}

\begin{frame}[fragile]{Con python}
\protect\hypertarget{con-python}{}
\begin{Shaded}
\begin{Highlighting}[]
\NormalTok{Omega}\OperatorTok{=}\BuiltInTok{set}\NormalTok{([}\DecValTok{1}\NormalTok{,}\DecValTok{2}\NormalTok{,}\DecValTok{3}\NormalTok{,}\DecValTok{4}\NormalTok{,}\DecValTok{5}\NormalTok{,}\DecValTok{6}\NormalTok{,}\DecValTok{7}\NormalTok{,}\DecValTok{8}\NormalTok{,}\DecValTok{9}\NormalTok{,}\DecValTok{10}\NormalTok{])}
\NormalTok{A}\OperatorTok{=}\BuiltInTok{set}\NormalTok{([}\DecValTok{1}\NormalTok{,}\DecValTok{2}\NormalTok{,}\DecValTok{3}\NormalTok{,}\DecValTok{4}\NormalTok{,}\DecValTok{5}\NormalTok{])}
\NormalTok{B}\OperatorTok{=}\BuiltInTok{set}\NormalTok{([}\DecValTok{1}\NormalTok{,}\DecValTok{4}\NormalTok{,}\DecValTok{5}\NormalTok{])}
\NormalTok{C}\OperatorTok{=}\BuiltInTok{set}\NormalTok{([}\DecValTok{4}\NormalTok{,}\DecValTok{6}\NormalTok{,}\DecValTok{7}\NormalTok{,}\DecValTok{8}\NormalTok{])}
\NormalTok{Omega}
\end{Highlighting}
\end{Shaded}

\begin{verbatim}
{1, 2, 3, 4, 5, 6, 7, 8, 9, 10}
\end{verbatim}
\end{frame}

\begin{frame}[fragile]{Con python}
\protect\hypertarget{con-python-1}{}
\begin{Shaded}
\begin{Highlighting}[]
\NormalTok{A}
\end{Highlighting}
\end{Shaded}

\begin{verbatim}
{1, 2, 3, 4, 5}
\end{verbatim}

\begin{Shaded}
\begin{Highlighting}[]
\NormalTok{B}
\end{Highlighting}
\end{Shaded}

\begin{verbatim}
{1, 4, 5}
\end{verbatim}

\begin{Shaded}
\begin{Highlighting}[]
\NormalTok{C}
\end{Highlighting}
\end{Shaded}

\begin{verbatim}
{8, 4, 6, 7}
\end{verbatim}
\end{frame}

\begin{frame}[fragile]{Con python}
\protect\hypertarget{con-python-2}{}
\begin{Shaded}
\begin{Highlighting}[]
\NormalTok{A }\OperatorTok{\&}\NormalTok{ B   }\CommentTok{\# intersección (\&: and/y)}
\end{Highlighting}
\end{Shaded}

\begin{verbatim}
{1, 4, 5}
\end{verbatim}

\begin{Shaded}
\begin{Highlighting}[]
\NormalTok{A }\OperatorTok{|}\NormalTok{ B   }\CommentTok{\# unión (|: or/o)}
\end{Highlighting}
\end{Shaded}

\begin{verbatim}
{1, 2, 3, 4, 5}
\end{verbatim}
\end{frame}

\begin{frame}[fragile]{Con python}
\protect\hypertarget{con-python-3}{}
\begin{Shaded}
\begin{Highlighting}[]
\NormalTok{A }\OperatorTok{{-}}\NormalTok{ C   }\CommentTok{\# diferencia }
\end{Highlighting}
\end{Shaded}

\begin{verbatim}
{1, 2, 3, 5}
\end{verbatim}

\begin{Shaded}
\begin{Highlighting}[]
\NormalTok{Omega}\OperatorTok{{-}}\NormalTok{C }\CommentTok{\# complementario.}
\end{Highlighting}
\end{Shaded}

\begin{verbatim}
{1, 2, 3, 5, 9, 10}
\end{verbatim}
\end{frame}

\hypertarget{combinatoria}{%
\section{Combinatoria}\label{combinatoria}}

\begin{frame}{Combinatoria. Introducción.}
\protect\hypertarget{combinatoria.-introducciuxf3n.}{}
La combinatoria es una rama de la matemática discreta que entre otras
cosas cuenta distintas configuraciones de objetos de un conjunto.

Por ejemplo si tenemos un equipo de baloncesto con 7 jugadores ¿cuántos
equipos de 5 jugadores distintos podemos formar?
\end{frame}

\begin{frame}{Combinatoria. Número binomial.}
\protect\hypertarget{combinatoria.-nuxfamero-binomial.}{}
\textbf{Número combinatorio o número binomial}

Nos da el número de subconjuntos de tamaño \(k\) de un conjunto de
tamaño \(n\). Este número es

\[
C_n^k={n\choose k} = \frac{n!}{k!\cdot (n-k)!}.
\]

Recordemos que \[
n!=1\cdot 2\cdot 3\cdots n.
\]
\end{frame}

\begin{frame}[fragile]{Combinatoria. Número binomial.}
\protect\hypertarget{combinatoria.-nuxfamero-binomial.-1}{}
En nuestro caso con 7 jugadores \(n=7\) el número de equipos distintos
de \(k=5\) es

\[
\begin{array}{rl}
C_7^5&={7\choose 5} = \frac{7!}{5!\cdot (7-5)!}=\frac{7!}{5!\cdot 2!} \\
&=\frac{1\cdot 2\cdot 3 \cdot 4\cdot 5\cdot 6\cdot 7}{1\cdot 2\cdot 3 \cdot 4\cdot 5\cdot 1\cdot 2}=\frac{6\cdot 7}{2}=\frac{42}{2}=21.
\end{array}
\]

Puedo formar 21 equipos distintos.

\textbf{Ejercicio}

Carga el paquete \texttt{gtools} de R y investiga la función
\texttt{combinations(n,\ r,\ v,\ set,\ repeats.allowed)} para calcular
todas las combinaciones anteriores.
\end{frame}

\begin{frame}{Combinatoria. Combinaciones con repetición}
\protect\hypertarget{combinatoria.-combinaciones-con-repeticiuxf3n}{}
En combinatoria, las combinaciones con repetición de un conjunto son las
distintas formas en que se puede hacer una selección de elementos de un
conjunto dado, permitiendo que las selecciones puedan repetirse.

El número \(CR_n^k\) de multiconjuntos con \(k\) elementos escogidos de
un conjunto con \(n\) elementos satisface:

\begin{itemize}
\tightlist
\item
  Es igual al número de combinaciones con repetición de \(k\) elementos
  escogidos de un conjunto con \(n\) elementos.
\item
  Es igual al número de formas de repartir \(k\) objetos en \(n\)
  grupos.
\end{itemize}

\[CR_n^k = \binom{n+k-1}{k} = \frac{(n+k-1)!}{k!(n-1)!}.\]
\end{frame}

\begin{frame}{Combinatoria. Combinaciones con repetición}
\protect\hypertarget{combinatoria.-combinaciones-con-repeticiuxf3n-1}{}
\textbf{Ejemplo}

Vamos a imaginar que vamos a repartir 12 caramelos entre Antonio,
Beatriz, Carlos y Dionisio (que representaremos como A, B, C, D). Una
posible forma de repartir los caramelos sería: dar 4 caramelos a
Antonio, 3 a Beatriz, 2 a Carlos y 3 a Dionisio. Dado que no importa el
orden en que se reparten, podemos representar esta selección como
AAAABBBCCDDD.

Otra forma posible de repartir los caramelos podría ser: dar 1 caramelo
a Antonio, ninguno a Beatriz y Carlos, los 11 restantes se los damos a
Dionisio. Esta repartición la representamos como ADDDDDDDDDDD
\end{frame}

\begin{frame}{Combinatoria. Combinaciones con repetición}
\protect\hypertarget{combinatoria.-combinaciones-con-repeticiuxf3n-2}{}
Recíprocamente, cualquier serie de 12 letras A, B, C, D se corresponde a
una forma de repartir los caramelos. Por ejemplo, la serie AAAABBBBBDDD
corresponde a: Dar 4 caramelos a Antonio, 5 caramelos a Beatriz, ninguno
a Carlos y 3 a Dionisio.

De esta forma, el número de formas de repartir los caramelos es:

\[CR_{4}^{12} = \binom{4+12-1}{4}\]
\end{frame}

\begin{frame}{Combinatoria. Variaciones.}
\protect\hypertarget{combinatoria.-variaciones.}{}
\textbf{Variaciones}

Con los número \(\{1,2,3\}\) ¿cuántos números de dos cifras distintas
podemos formar sin repetir ninguna cifra?

La podemos escribir

\[12,13,21,23,31,32\]

Luego hay seis casos
\end{frame}

\begin{frame}{Combinatoria. Variaciones (sin repetición).}
\protect\hypertarget{combinatoria.-variaciones-sin-repeticiuxf3n.}{}
Denotaremos las variaciones (sin repetición) de \(k\) elementos (de
orden \(k\)) de un conjunto de \(n\) elementos por \(V_n^k\) su valor es

\[
V_n^k=\frac{n!}{(n-k)!}=(n-k+1)\cdot (n-k+2)\cdots n.
\]
\end{frame}

\begin{frame}[fragile]{Combinatoria. Variaciones.}
\protect\hypertarget{combinatoria.-variaciones.-1}{}
En nuestro ejemplo con \(n=3\) dígitos podemos escribir las siguientes
variaciones de orden \(k=2\)

\[
V^{k=2}_{n=3}=\frac{3!}{(3-2)!}=\frac{1\cdot 2\cdot 3}{1}=6.
\]

\textbf{Ejercicio}

Carga el paquete \texttt{gtools} de R y investiga la función
\texttt{permutations(n,\ r,\ v,\ set,\ repeats.allowed)} para calcular
todas las variaciones anteriores.
\end{frame}

\begin{frame}{Combinatoria. Variaciones con repetición.}
\protect\hypertarget{combinatoria.-variaciones-con-repeticiuxf3n.}{}
\textbf{Variaciones con repetición}

¿Y repitiendo algún dígito?

\[VR_n^k=n^k\]

Efectivamente en nuestro caso

\[11,12,13,21,22,23,31,32,33\]

\[
VR^{k=2}_{n=3}=n^k.
\]
\end{frame}

\begin{frame}{Permutaciones}
\protect\hypertarget{permutaciones}{}
Las permutaciones de un conjunto de cardinal \(n\) son todas las
variaciones de orden máximo \(n\). Las denotamos y valen:

\[
P_n=V_n^n=n!
\]
\end{frame}

\begin{frame}[fragile]{Permutaciones}
\protect\hypertarget{permutaciones-1}{}
Por ejemplo todos los números que se pueden escribir ordenando todos los
dígitos \(\{1,2,3\}\) sin repetir ninguno

\begin{Shaded}
\begin{Highlighting}[]
\FunctionTok{library}\NormalTok{(combinat)}
\ControlFlowTok{for}\NormalTok{(permutacion }\ControlFlowTok{in} \FunctionTok{permn}\NormalTok{(}\DecValTok{3}\NormalTok{)) }\FunctionTok{print}\NormalTok{(permutacion)}
\end{Highlighting}
\end{Shaded}

\begin{verbatim}
[1] 1 2 3
[1] 1 3 2
[1] 3 1 2
[1] 3 2 1
[1] 2 3 1
[1] 2 1 3
\end{verbatim}

Efectivamente \[
P_3=3!=1\cdot  2\cdot 3.
\]
\end{frame}

\begin{frame}[fragile]{Permutaciones}
\protect\hypertarget{permutaciones-2}{}
\textbf{Ejercicio}

Carga el paquete \texttt{combinat} de R e investiga la funcion
\texttt{permn} para calcular todas las permutaciones anteriores.

\textbf{Ejercicio}

Investiga el paquete \texttt{itertools} y la función \texttt{comb} de
\texttt{scipy.misc} de Python e investiga sus funciones para todas las
formas de contar que hemos visto en este tema.

\textbf{Ejercicio}

La función gamma de Euler, cobrará mucha importancia en el curso de
estadística. Comprueba que la función \texttt{gamma(x+1)} da el mismo
valor que la función \texttt{factorial(x)} en \texttt{R} para todo
\(x = \{1,2,3,\ldots,10\}\).
\end{frame}

\begin{frame}{Números multinomiales. Permutaciones con repetición.}
\protect\hypertarget{nuxfameros-multinomiales.-permutaciones-con-repeticiuxf3n.}{}
Consideremos un conjunto de elementos \(\{a_1, a_2, \ldots, a_k\}\).

Entoces, si cada uno de los objetos \(a_i\) de un conjunto, aparece
repetido \(n_i\) veces para cada \(i\) desde 1 hasta \(k\), entonces el
número de permutaciones con elementos repetidos es:

\[PR_n^{n_1,n_2,\ldots,n_k} = {{n}\choose {n_1\quad n_2 \quad\ldots \quad n_k}}=\frac{n!}{n_1!\cdot n_2!\cdot \ldots \cdot n_k!},\]
donde \(n=n_1+n_2+\cdots+n_k\).
\end{frame}

\begin{frame}[fragile]{Números multinomiales. Permutaciones con
repetición.}
\protect\hypertarget{nuxfameros-multinomiales.-permutaciones-con-repeticiuxf3n.-1}{}
\textbf{Ejemplo}

¿Cuantas palabras diferentes se pueden formar con las letras de la
palabra \texttt{PROBABILIDAD}?

El conjunto de letras de la palabra considerada es el siguiente:
\(\{A, B, D, I, L, O, P, R\}\) con las repeticiones siguientes: las
letras A, B, D, e I, aparecen 2 veces cada una; y las letras L, O, P, R
una vez cada una de ellas.

Por tanto, utilizando la fórmula anterior, tenemos que el número de
palabras (permutaciones con elementos repetidos) que podemos formar es

\[PR^{2,2,2,2,1,1,1,1}_{12} = \frac{12!}{(2!)^4(1!)^4} = 29937600.\]
\end{frame}

\hypertarget{para-acabar}{%
\section{Para acabar}\label{para-acabar}}

\begin{frame}{Principios básicos para contar cardinales de conjuntos}
\protect\hypertarget{principios-buxe1sicos-para-contar-cardinales-de-conjuntos}{}
\textbf{El principio de la suma}

Sean \(A_1, A_2,\ldots, A_n\) conjuntos disjuntos dos a dos, es decir
\(A_i\cap A_j=\emptyset\) para todo \(i\not= j\), \(i,j=1,2,\ldots n\).
Entonces

\[\#(\cup_{i=1}^n A_i)=\sum_{i=1}^n \#(A_i).\]
\end{frame}

\begin{frame}{Principios básicos para contar cardinales de conjuntos}
\protect\hypertarget{principios-buxe1sicos-para-contar-cardinales-de-conjuntos-1}{}
\textbf{Principio de unión exclusión}

Consideremos dos conjuntos cualesquiera \(A_1, A_2\) entonces el
cardinal de su unión es

\[\#(A_1\cup A_2)=\#(A_1)+\#(A_2)-\#(A_1\cap A_2).\]
\end{frame}

\begin{frame}{Principios básicos para contar cardinales de conjuntos}
\protect\hypertarget{principios-buxe1sicos-para-contar-cardinales-de-conjuntos-2}{}
\textbf{El principio del producto}

Sean \(A_1,A_2,\ldots A_n\)

\[
\begin{array}{ll}
\#(A_1\times A_2\times \cdots A_n)=&\#\left(\{(a_1,a_2,\ldots a_n)| a_i\in A_i, i=1,2,\ldots n\}\right)\\
&=\prod_{i=1}^n \#(A_i).
\end{array}
\]
\end{frame}

\begin{frame}{Otros aspectos a tener en cuenta}
\protect\hypertarget{otros-aspectos-a-tener-en-cuenta}{}
Evidentemente nos hemos dejado muchas otras propiedades básicas de
teoría de conjuntos y de combinatoria como:

\begin{itemize}
\tightlist
\item
  Propiedades de los números combinatorios.
\item
  Binomio de Newton.
\item
  Multinomio de Newton.
\end{itemize}

Si nos son necesarias las volveremos a repetir a lo largo del curso o
bien daremos enlaces para que las podáis estudiar en paralelo.
\end{frame}



\end{document}

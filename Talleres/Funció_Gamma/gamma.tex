% Options for packages loaded elsewhere
\PassOptionsToPackage{unicode}{hyperref}
\PassOptionsToPackage{hyphens}{url}
\PassOptionsToPackage{dvipsnames,svgnames,x11names}{xcolor}
%
\documentclass[
]{article}
\usepackage{amsmath,amssymb}
\usepackage{lmodern}
\usepackage{iftex}
\ifPDFTeX
  \usepackage[T1]{fontenc}
  \usepackage[utf8]{inputenc}
  \usepackage{textcomp} % provide euro and other symbols
\else % if luatex or xetex
  \usepackage{unicode-math}
  \defaultfontfeatures{Scale=MatchLowercase}
  \defaultfontfeatures[\rmfamily]{Ligatures=TeX,Scale=1}
\fi
% Use upquote if available, for straight quotes in verbatim environments
\IfFileExists{upquote.sty}{\usepackage{upquote}}{}
\IfFileExists{microtype.sty}{% use microtype if available
  \usepackage[]{microtype}
  \UseMicrotypeSet[protrusion]{basicmath} % disable protrusion for tt fonts
}{}
\makeatletter
\@ifundefined{KOMAClassName}{% if non-KOMA class
  \IfFileExists{parskip.sty}{%
    \usepackage{parskip}
  }{% else
    \setlength{\parindent}{0pt}
    \setlength{\parskip}{6pt plus 2pt minus 1pt}}
}{% if KOMA class
  \KOMAoptions{parskip=half}}
\makeatother
\usepackage{xcolor}
\usepackage[margin=1in]{geometry}
\usepackage{color}
\usepackage{fancyvrb}
\newcommand{\VerbBar}{|}
\newcommand{\VERB}{\Verb[commandchars=\\\{\}]}
\DefineVerbatimEnvironment{Highlighting}{Verbatim}{commandchars=\\\{\}}
% Add ',fontsize=\small' for more characters per line
\usepackage{framed}
\definecolor{shadecolor}{RGB}{248,248,248}
\newenvironment{Shaded}{\begin{snugshade}}{\end{snugshade}}
\newcommand{\AlertTok}[1]{\textcolor[rgb]{0.94,0.16,0.16}{#1}}
\newcommand{\AnnotationTok}[1]{\textcolor[rgb]{0.56,0.35,0.01}{\textbf{\textit{#1}}}}
\newcommand{\AttributeTok}[1]{\textcolor[rgb]{0.77,0.63,0.00}{#1}}
\newcommand{\BaseNTok}[1]{\textcolor[rgb]{0.00,0.00,0.81}{#1}}
\newcommand{\BuiltInTok}[1]{#1}
\newcommand{\CharTok}[1]{\textcolor[rgb]{0.31,0.60,0.02}{#1}}
\newcommand{\CommentTok}[1]{\textcolor[rgb]{0.56,0.35,0.01}{\textit{#1}}}
\newcommand{\CommentVarTok}[1]{\textcolor[rgb]{0.56,0.35,0.01}{\textbf{\textit{#1}}}}
\newcommand{\ConstantTok}[1]{\textcolor[rgb]{0.00,0.00,0.00}{#1}}
\newcommand{\ControlFlowTok}[1]{\textcolor[rgb]{0.13,0.29,0.53}{\textbf{#1}}}
\newcommand{\DataTypeTok}[1]{\textcolor[rgb]{0.13,0.29,0.53}{#1}}
\newcommand{\DecValTok}[1]{\textcolor[rgb]{0.00,0.00,0.81}{#1}}
\newcommand{\DocumentationTok}[1]{\textcolor[rgb]{0.56,0.35,0.01}{\textbf{\textit{#1}}}}
\newcommand{\ErrorTok}[1]{\textcolor[rgb]{0.64,0.00,0.00}{\textbf{#1}}}
\newcommand{\ExtensionTok}[1]{#1}
\newcommand{\FloatTok}[1]{\textcolor[rgb]{0.00,0.00,0.81}{#1}}
\newcommand{\FunctionTok}[1]{\textcolor[rgb]{0.00,0.00,0.00}{#1}}
\newcommand{\ImportTok}[1]{#1}
\newcommand{\InformationTok}[1]{\textcolor[rgb]{0.56,0.35,0.01}{\textbf{\textit{#1}}}}
\newcommand{\KeywordTok}[1]{\textcolor[rgb]{0.13,0.29,0.53}{\textbf{#1}}}
\newcommand{\NormalTok}[1]{#1}
\newcommand{\OperatorTok}[1]{\textcolor[rgb]{0.81,0.36,0.00}{\textbf{#1}}}
\newcommand{\OtherTok}[1]{\textcolor[rgb]{0.56,0.35,0.01}{#1}}
\newcommand{\PreprocessorTok}[1]{\textcolor[rgb]{0.56,0.35,0.01}{\textit{#1}}}
\newcommand{\RegionMarkerTok}[1]{#1}
\newcommand{\SpecialCharTok}[1]{\textcolor[rgb]{0.00,0.00,0.00}{#1}}
\newcommand{\SpecialStringTok}[1]{\textcolor[rgb]{0.31,0.60,0.02}{#1}}
\newcommand{\StringTok}[1]{\textcolor[rgb]{0.31,0.60,0.02}{#1}}
\newcommand{\VariableTok}[1]{\textcolor[rgb]{0.00,0.00,0.00}{#1}}
\newcommand{\VerbatimStringTok}[1]{\textcolor[rgb]{0.31,0.60,0.02}{#1}}
\newcommand{\WarningTok}[1]{\textcolor[rgb]{0.56,0.35,0.01}{\textbf{\textit{#1}}}}
\usepackage{graphicx}
\makeatletter
\def\maxwidth{\ifdim\Gin@nat@width>\linewidth\linewidth\else\Gin@nat@width\fi}
\def\maxheight{\ifdim\Gin@nat@height>\textheight\textheight\else\Gin@nat@height\fi}
\makeatother
% Scale images if necessary, so that they will not overflow the page
% margins by default, and it is still possible to overwrite the defaults
% using explicit options in \includegraphics[width, height, ...]{}
\setkeys{Gin}{width=\maxwidth,height=\maxheight,keepaspectratio}
% Set default figure placement to htbp
\makeatletter
\def\fps@figure{htbp}
\makeatother
\setlength{\emergencystretch}{3em} % prevent overfull lines
\providecommand{\tightlist}{%
  \setlength{\itemsep}{0pt}\setlength{\parskip}{0pt}}
\setcounter{secnumdepth}{5}
\renewcommand{\contentsname}{Contenidos}
\ifLuaTeX
  \usepackage{selnolig}  % disable illegal ligatures
\fi
\IfFileExists{bookmark.sty}{\usepackage{bookmark}}{\usepackage{hyperref}}
\IfFileExists{xurl.sty}{\usepackage{xurl}}{} % add URL line breaks if available
\urlstyle{same} % disable monospaced font for URLs
\hypersetup{
  pdftitle={Taller función gamma},
  pdfauthor={Estadística},
  colorlinks=true,
  linkcolor={red},
  filecolor={Maroon},
  citecolor={blue},
  urlcolor={blue},
  pdfcreator={LaTeX via pandoc}}

\title{Taller función gamma}
\author{Estadística}
\date{15 septiembre, 2022}

\begin{document}
\maketitle

{
\hypersetup{linkcolor=blue}
\setcounter{tocdepth}{2}
\tableofcontents
}
\hypertarget{lab2}{%
\section{Lab2}\label{lab2}}

En esta asignatura entrenaremos \emph{cosas} de matemáticas y sus
amigos.

\hypertarget{la-funciuxf3n-gamma}{%
\subsection{La función Gamma}\label{la-funciuxf3n-gamma}}

La función Gamma \(\Gamma\) tiene diversas definiciones en la
matemática. La definición que utilizaremos es:

\[ \Gamma(z)= \int_0^{\infty} x^{z-1} \cdot e^{-x}dx, \mbox{ donde } z\in \mathbb{R}.\]

Resolvamos esta integral en el caso \(\Gamma(z+1)\) con
\(z\in \mathbb{R}\). Recordemos que la fórmula integración por partes en
este caso es:

\[\int_{0}^{\infty} u \cdot d v =\left[u\cdot v \right]_0^\infty-\int_0^{\infty} v \cdot du.\]

Apliquemos el método de integración por partes a la función \(\Gamma\)

\[
\begin{aligned}
\Gamma(z+1) &=  \int_0^{\infty} x^{z} \cdot e^{-x}dx= 
\left|
\begin{matrix} u=x^{z}  & dv= e^{-x}\cdot  dx 
\\ du= z \cdot x^(z+1)  & v=-e^{-x} 
\end{matrix}
\right|
\\
&=\left[-x^z\cdot e^{-x}\right]_0^\infty
-\int_0^{\infty} z\cdot x^{z-1} \cdot \left(-e^{-x}\right)\cdot  dx
\\
&=
\lim_{x\to\infty}\left(-x^z\cdot e^{-x}\right)
+
z\cdot \int_0^{\infty} x^{z-1} \cdot e^{-x}\cdot dx.
\end{aligned}
\]

como

\[\lim_{x\to\infty}\left(-x^z\cdot e^{-x}\right)=0\]

tenemos que

\[\Gamma(z+1)= 
z\cdot \int_0^{\infty} x^{z-1} \cdot e^{-x}\cdot  dx.
\]

Por lo que hemos encontrado una fórmula recurrente, en al que si
queremos saber \(\Gamma(z+1)\) tenemos que saber que vale \(\Gamma(z)\)
y utilizar la fórmula anterior.

\hypertarget{las-fuxf3rmulas-recursivas}{%
\subsection{Las fórmulas recursivas}\label{las-fuxf3rmulas-recursivas}}

La fórmulas recursivas son las que dependen de un valor anterior al que
se calcula. La más popular es el factorial

La definición de factorial de un un número natural \(n\in\mathbb{N}\),
es ¡¡obviamente!! recursiva

\[
\begin{aligned}
\verb+factorial+ = n!: & \mathbb{N} \longrightarrow  \mathbb{N}\\
& n \longrightarrow  n\cdot (n-1)\cdot (n-2) \ldots \cdot 3\cdot 2\cdot 1.
\end{aligned}
\]

Se define con estas reglas:

\begin{enumerate}
\def\labelenumi{\arabic{enumi}.}
\tightlist
\item
  \texttt{factorial(0)}=\(0!=1\).
\item
  \texttt{factorial(n+1)=}\((n+1)!=(n+1)*factorial(n).\)
\end{enumerate}

En la notación matemática, como ya sabéis el factorial se representa con
el símbolo de exclamación/admiración; así

\begin{enumerate}
\def\labelenumi{\arabic{enumi}.}
\tightlist
\item
  \texttt{factorial(0)}:= \(0!=1\).
\item
  \texttt{factorial(n+1)}:= \((n+1)!=(n+1)\cdot n!\).
\end{enumerate}

Así tenemos que

\begin{itemize}
\tightlist
\item
  \(0!=1.\)
\item
  \(1!=1.\)
\item
  \(2!= 2\cdot 1= 2.\)
\item
  \(3!=3\cdot 2\cdot 1= 6.\)
\item
  \(4!=4\cdot 3\cdot 2\cdot 1 =24.\)
\item
  \(\ldots \ldots\)
\item
  \(n!= n\cdot (n-1) \cdot (n-2) \cdots 3\cdot 2\cdot 1.\)
\item
  \((n+1)!= (n+1)\cdot n!= (n+1)\cdot n\cdot (n-1) \cdot (n-2) \cdots 3\cdot 2\cdot 1.\)
\end{itemize}

\hypertarget{gruxe1fica-funciuxf3n-gamma-en-los-reales}{%
\section{Gráfica función Gamma en los
reales}\label{gruxe1fica-funciuxf3n-gamma-en-los-reales}}

\begin{Shaded}
\begin{Highlighting}[]
\FunctionTok{curve}\NormalTok{(}\FunctionTok{gamma}\NormalTok{(x),}\AttributeTok{xlim=}\FunctionTok{c}\NormalTok{(}\DecValTok{1}\NormalTok{,}\DecValTok{10}\NormalTok{),}\AttributeTok{col=}\StringTok{"red"}\NormalTok{,}\AttributeTok{ylab=}\StringTok{"Gamma(x)"}\NormalTok{,}\AttributeTok{lwd=}\DecValTok{2}\NormalTok{,}\AttributeTok{frame.plot=}\ConstantTok{TRUE}\NormalTok{,}\AttributeTok{main=}\StringTok{"Función Gamma"}\NormalTok{)}
\end{Highlighting}
\end{Shaded}

\includegraphics{gamma_files/figure-latex/unnamed-chunk-1-1.pdf}

\end{document}

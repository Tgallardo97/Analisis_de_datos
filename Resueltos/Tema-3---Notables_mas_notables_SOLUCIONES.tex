% Options for packages loaded elsewhere
\PassOptionsToPackage{unicode}{hyperref}
\PassOptionsToPackage{hyphens}{url}
\PassOptionsToPackage{dvipsnames,svgnames,x11names}{xcolor}
%
\documentclass[
]{article}
\usepackage{amsmath,amssymb}
\usepackage{lmodern}
\usepackage{iftex}
\ifPDFTeX
  \usepackage[T1]{fontenc}
  \usepackage[utf8]{inputenc}
  \usepackage{textcomp} % provide euro and other symbols
\else % if luatex or xetex
  \usepackage{unicode-math}
  \defaultfontfeatures{Scale=MatchLowercase}
  \defaultfontfeatures[\rmfamily]{Ligatures=TeX,Scale=1}
\fi
% Use upquote if available, for straight quotes in verbatim environments
\IfFileExists{upquote.sty}{\usepackage{upquote}}{}
\IfFileExists{microtype.sty}{% use microtype if available
  \usepackage[]{microtype}
  \UseMicrotypeSet[protrusion]{basicmath} % disable protrusion for tt fonts
}{}
\makeatletter
\@ifundefined{KOMAClassName}{% if non-KOMA class
  \IfFileExists{parskip.sty}{%
    \usepackage{parskip}
  }{% else
    \setlength{\parindent}{0pt}
    \setlength{\parskip}{6pt plus 2pt minus 1pt}}
}{% if KOMA class
  \KOMAoptions{parskip=half}}
\makeatother
\usepackage{xcolor}
\usepackage[margin=1in]{geometry}
\usepackage{color}
\usepackage{fancyvrb}
\newcommand{\VerbBar}{|}
\newcommand{\VERB}{\Verb[commandchars=\\\{\}]}
\DefineVerbatimEnvironment{Highlighting}{Verbatim}{commandchars=\\\{\}}
% Add ',fontsize=\small' for more characters per line
\usepackage{framed}
\definecolor{shadecolor}{RGB}{248,248,248}
\newenvironment{Shaded}{\begin{snugshade}}{\end{snugshade}}
\newcommand{\AlertTok}[1]{\textcolor[rgb]{0.94,0.16,0.16}{#1}}
\newcommand{\AnnotationTok}[1]{\textcolor[rgb]{0.56,0.35,0.01}{\textbf{\textit{#1}}}}
\newcommand{\AttributeTok}[1]{\textcolor[rgb]{0.77,0.63,0.00}{#1}}
\newcommand{\BaseNTok}[1]{\textcolor[rgb]{0.00,0.00,0.81}{#1}}
\newcommand{\BuiltInTok}[1]{#1}
\newcommand{\CharTok}[1]{\textcolor[rgb]{0.31,0.60,0.02}{#1}}
\newcommand{\CommentTok}[1]{\textcolor[rgb]{0.56,0.35,0.01}{\textit{#1}}}
\newcommand{\CommentVarTok}[1]{\textcolor[rgb]{0.56,0.35,0.01}{\textbf{\textit{#1}}}}
\newcommand{\ConstantTok}[1]{\textcolor[rgb]{0.00,0.00,0.00}{#1}}
\newcommand{\ControlFlowTok}[1]{\textcolor[rgb]{0.13,0.29,0.53}{\textbf{#1}}}
\newcommand{\DataTypeTok}[1]{\textcolor[rgb]{0.13,0.29,0.53}{#1}}
\newcommand{\DecValTok}[1]{\textcolor[rgb]{0.00,0.00,0.81}{#1}}
\newcommand{\DocumentationTok}[1]{\textcolor[rgb]{0.56,0.35,0.01}{\textbf{\textit{#1}}}}
\newcommand{\ErrorTok}[1]{\textcolor[rgb]{0.64,0.00,0.00}{\textbf{#1}}}
\newcommand{\ExtensionTok}[1]{#1}
\newcommand{\FloatTok}[1]{\textcolor[rgb]{0.00,0.00,0.81}{#1}}
\newcommand{\FunctionTok}[1]{\textcolor[rgb]{0.00,0.00,0.00}{#1}}
\newcommand{\ImportTok}[1]{#1}
\newcommand{\InformationTok}[1]{\textcolor[rgb]{0.56,0.35,0.01}{\textbf{\textit{#1}}}}
\newcommand{\KeywordTok}[1]{\textcolor[rgb]{0.13,0.29,0.53}{\textbf{#1}}}
\newcommand{\NormalTok}[1]{#1}
\newcommand{\OperatorTok}[1]{\textcolor[rgb]{0.81,0.36,0.00}{\textbf{#1}}}
\newcommand{\OtherTok}[1]{\textcolor[rgb]{0.56,0.35,0.01}{#1}}
\newcommand{\PreprocessorTok}[1]{\textcolor[rgb]{0.56,0.35,0.01}{\textit{#1}}}
\newcommand{\RegionMarkerTok}[1]{#1}
\newcommand{\SpecialCharTok}[1]{\textcolor[rgb]{0.00,0.00,0.00}{#1}}
\newcommand{\SpecialStringTok}[1]{\textcolor[rgb]{0.31,0.60,0.02}{#1}}
\newcommand{\StringTok}[1]{\textcolor[rgb]{0.31,0.60,0.02}{#1}}
\newcommand{\VariableTok}[1]{\textcolor[rgb]{0.00,0.00,0.00}{#1}}
\newcommand{\VerbatimStringTok}[1]{\textcolor[rgb]{0.31,0.60,0.02}{#1}}
\newcommand{\WarningTok}[1]{\textcolor[rgb]{0.56,0.35,0.01}{\textbf{\textit{#1}}}}
\usepackage{graphicx}
\makeatletter
\def\maxwidth{\ifdim\Gin@nat@width>\linewidth\linewidth\else\Gin@nat@width\fi}
\def\maxheight{\ifdim\Gin@nat@height>\textheight\textheight\else\Gin@nat@height\fi}
\makeatother
% Scale images if necessary, so that they will not overflow the page
% margins by default, and it is still possible to overwrite the defaults
% using explicit options in \includegraphics[width, height, ...]{}
\setkeys{Gin}{width=\maxwidth,height=\maxheight,keepaspectratio}
% Set default figure placement to htbp
\makeatletter
\def\fps@figure{htbp}
\makeatother
\setlength{\emergencystretch}{3em} % prevent overfull lines
\providecommand{\tightlist}{%
  \setlength{\itemsep}{0pt}\setlength{\parskip}{0pt}}
\setcounter{secnumdepth}{5}
\ifLuaTeX
  \usepackage{selnolig}  % disable illegal ligatures
\fi
\IfFileExists{bookmark.sty}{\usepackage{bookmark}}{\usepackage{hyperref}}
\IfFileExists{xurl.sty}{\usepackage{xurl}}{} % add URL line breaks if available
\urlstyle{same} % disable monospaced font for URLs
\hypersetup{
  pdftitle={SOLUCIONES: Ejercicios Tema 3 - Distribuciones Notables: más distribuciones notables},
  pdfauthor={Otras distribuciones notables},
  colorlinks=true,
  linkcolor={Maroon},
  filecolor={Maroon},
  citecolor={Blue},
  urlcolor={Blue},
  pdfcreator={LaTeX via pandoc}}

\title{SOLUCIONES: Ejercicios Tema 3 - Distribuciones Notables: más
distribuciones notables}
\author{Otras distribuciones notables}
\date{22 mayo, 2023}

\begin{document}
\maketitle

{
\hypersetup{linkcolor=blue}
\setcounter{tocdepth}{2}
\tableofcontents
}
\hypertarget{distribuciones-notables-muxe1s-distribuciones-notables.}{%
\section{Distribuciones Notables: más distribuciones
notables.}\label{distribuciones-notables-muxe1s-distribuciones-notables.}}

\hypertarget{ley-de-bendford}{%
\subsection{\texorpdfstring{\textbf{Ley de
Bendford}}{Ley de Bendford}}\label{ley-de-bendford}}

La ley de Benford es una curiosa distribución de probabilidad que suele
aparecer en la distribución del primer dígito de las cantidades
registradas en contabilidades y en observaciones científicas o datos
numéricos. La variable \(X\) sigue una distribución discreta Benford con
dominio \(D_X=\{1,2,3,4,5,7,8,9\}\) son 9 dígitos (se elimina el cero) y
su función de probabilidad es (donde \(\log\) es el logaritmo en base
10) \[
P_X(x)=P(X=x)=\log(x+1)-\log(x).
\]

\begin{itemize}
\item
  \begin{enumerate}
  \def\labelenumi{\alph{enumi})}
  \tightlist
  \item
    Calcular la media y la varianza de \(X\).
  \end{enumerate}
\item
  \begin{enumerate}
  \def\labelenumi{\alph{enumi})}
  \setcounter{enumi}{1}
  \tightlist
  \item
    Calcular la función de distribución de \(X\).
  \end{enumerate}
\item
  \begin{enumerate}
  \def\labelenumi{\alph{enumi})}
  \setcounter{enumi}{2}
  \tightlist
  \item
    ¿Cuál es el dígito más frecuente (moda)?
  \end{enumerate}
\item
  \begin{enumerate}
  \def\labelenumi{\alph{enumi})}
  \setcounter{enumi}{3}
  \tightlist
  \item
    Construid con R las funciones de probabilidad y de distribución de
    \(X\).
  \end{enumerate}
\item
  \begin{enumerate}
  \def\labelenumi{\alph{enumi})}
  \setcounter{enumi}{4}
  \tightlist
  \item
    Dibujar con R las funciones del apartado anterior.
  \end{enumerate}
\end{itemize}

\hypertarget{soluciuxf3n}{%
\subsubsection{Solución}\label{soluciuxf3n}}

\begin{enumerate}
\def\labelenumi{\alph{enumi})}
\tightlist
\item
  Recordad que en R \texttt{log10} es el logartimo en base 10
\end{enumerate}

Como \(\log(d+1)-\log(d)=\log(\frac{d+1}{d})\). Podemos implementar la
función de probabilidad de Bendford con el siguiente código R

\begin{Shaded}
\begin{Highlighting}[]
\NormalTok{dBendford }\OtherTok{=} \ControlFlowTok{function}\NormalTok{(x)\{}
  \FunctionTok{sapply}\NormalTok{(x, }\AttributeTok{FUN=}\ControlFlowTok{function}\NormalTok{(x1)}
\NormalTok{  \{}
    \ControlFlowTok{if}\NormalTok{ (x1 }\SpecialCharTok{\%in\%} \DecValTok{1}\SpecialCharTok{:}\DecValTok{9}\NormalTok{)}
\NormalTok{    \{}\FunctionTok{log10}\NormalTok{(x1}\SpecialCharTok{+}\DecValTok{1}\NormalTok{)}\SpecialCharTok{{-}}\FunctionTok{log10}\NormalTok{(x1)\}}
    \ControlFlowTok{else}\NormalTok{\{}\DecValTok{0}\NormalTok{\}}
\NormalTok{  \})}
\NormalTok{\}}

\FunctionTok{dBendford}\NormalTok{(}\DecValTok{1}\SpecialCharTok{:}\DecValTok{9}\NormalTok{)}
\end{Highlighting}
\end{Shaded}

\begin{verbatim}
## [1] 0.30103000 0.17609126 0.12493874 0.09691001 0.07918125 0.06694679 0.05799195
## [8] 0.05115252 0.04575749
\end{verbatim}

\begin{Shaded}
\begin{Highlighting}[]
\FunctionTok{sum}\NormalTok{(}\FunctionTok{dBendford}\NormalTok{(}\DecValTok{1}\SpecialCharTok{:}\DecValTok{9}\NormalTok{))}
\end{Highlighting}
\end{Shaded}

\begin{verbatim}
## [1] 1
\end{verbatim}

Así la media \(\mu\) será

\begin{Shaded}
\begin{Highlighting}[]
\NormalTok{mu}\OtherTok{=}\FunctionTok{sum}\NormalTok{(}\FunctionTok{c}\NormalTok{(}\DecValTok{1}\SpecialCharTok{:}\DecValTok{9}\NormalTok{)}\SpecialCharTok{*}\FunctionTok{dBendford}\NormalTok{(}\DecValTok{1}\SpecialCharTok{:}\DecValTok{9}\NormalTok{))}
\NormalTok{mu}
\end{Highlighting}
\end{Shaded}

\begin{verbatim}
## [1] 3.440237
\end{verbatim}

\begin{Shaded}
\begin{Highlighting}[]
\NormalTok{sumx2}\OtherTok{=}\FunctionTok{sum}\NormalTok{(}\FunctionTok{c}\NormalTok{(}\DecValTok{1}\SpecialCharTok{:}\DecValTok{9}\NormalTok{)}\SpecialCharTok{\^{}}\DecValTok{2}\SpecialCharTok{*}\FunctionTok{dBendford}\NormalTok{(}\DecValTok{1}\SpecialCharTok{:}\DecValTok{9}\NormalTok{))}
\NormalTok{sumx2}
\end{Highlighting}
\end{Shaded}

\begin{verbatim}
## [1] 17.89174
\end{verbatim}

\begin{Shaded}
\begin{Highlighting}[]
\NormalTok{sigma2}\OtherTok{=}\NormalTok{sumx2}\SpecialCharTok{{-}}\NormalTok{mu}\SpecialCharTok{\^{}}\DecValTok{2}
\NormalTok{sigma2}
\end{Highlighting}
\end{Shaded}

\begin{verbatim}
## [1] 6.056513
\end{verbatim}

\begin{Shaded}
\begin{Highlighting}[]
\NormalTok{sigma}\OtherTok{=}\FunctionTok{sqrt}\NormalTok{(sigma2)}
\NormalTok{sigma}
\end{Highlighting}
\end{Shaded}

\begin{verbatim}
## [1] 2.460998
\end{verbatim}

En resumen La variable de Bendford tiene dominio
\(D_X=\{1,2,3,4,5,6,7,8,9\}\) y función de probabilidad

\[
P_X(x)=P(X=x)=\left\{
\begin{array}{ll}
0.30103 &  \mbox{si } x=1\\
0.1760913 &  \mbox{si } x=2\\
0.1249387 &  \mbox{si } x=3\\
0.09691 &  \mbox{si } x=4\\
0.0791812 &  \mbox{si } x=5\\
0.0669468 &  \mbox{si } x=6\\
0.0579919 &  \mbox{si } x=7\\
0.0511525 &  \mbox{si } x=8\\
0.0457575 &  \mbox{si } x=9\\
0 & \mbox{ en otro caso}
\end{array}
\right.
\]

\begin{eqnarray*}
E(X) &=& \sum_{k=1}^9 x\cdot P_X(x)= 1 \cdot 0.30103+ 2 \cdot 0.1760913+3 \cdot 0.1249387+4 \cdot 0.09691+5 \cdot 0.0791812\\
     &+&  6 \cdot 0.0669468+ 7 \cdot 0.0579919+ 8 \cdot 0.0511525+9 \cdot 0.0457575\\
     &=& 3.440237
\end{eqnarray*}

\begin{eqnarray*}
E(X^2) &=& \sum_{k=1}^9 x^2\cdot P_X(x)= 1 \cdot 0.30103+ 4 \cdot 0.1760913+9 \cdot 0.1249387+ 16 \cdot 0.09691+25 \cdot 0.0791812\\
     &+&  36 \cdot 0.0669468+ 49 \cdot 0.0579919+ 64 \cdot 0.0511525+81 \cdot 0.0457575\\
     &=& 3.440237
\end{eqnarray*}

Y por último \(Var(X)=E(X^2)-(E(X))^2=17.891743-(3.440237)^2=6.0565126\)
y la desviación típica es \(\sqrt{Var(X)}=2.4609983.\)

\begin{enumerate}
\def\labelenumi{\alph{enumi})}
\setcounter{enumi}{1}
\tightlist
\item
  Ahora nos piden \(F_X(x)=P(X\leq x)\). Con R es
\end{enumerate}

\begin{Shaded}
\begin{Highlighting}[]
\NormalTok{pBendford}\OtherTok{=}\ControlFlowTok{function}\NormalTok{(x)\{}
  \FunctionTok{sapply}\NormalTok{(x,}\AttributeTok{FUN=}\ControlFlowTok{function}\NormalTok{(x)\{}
\NormalTok{  probs}\OtherTok{=}\FunctionTok{cumsum}\NormalTok{(}\FunctionTok{dBendford}\NormalTok{(}\DecValTok{1}\SpecialCharTok{:}\DecValTok{9}\NormalTok{))}
\NormalTok{  xfloor}\OtherTok{=}\FunctionTok{floor}\NormalTok{(x)}
  \ControlFlowTok{if}\NormalTok{(xfloor}\SpecialCharTok{\textless{}}\DecValTok{1}\NormalTok{)\{}\DecValTok{0}\NormalTok{\} }\ControlFlowTok{else}\NormalTok{ \{}\ControlFlowTok{if}\NormalTok{(xfloor}\SpecialCharTok{\textgreater{}}\DecValTok{8}\NormalTok{) \{}\DecValTok{1}\NormalTok{\} }\ControlFlowTok{else}\NormalTok{ \{probs[xfloor]\}\}}
\NormalTok{\})}
\NormalTok{\}}

\FunctionTok{pBendford}\NormalTok{(}\DecValTok{0}\SpecialCharTok{:}\DecValTok{9}\NormalTok{)}
\end{Highlighting}
\end{Shaded}

\begin{verbatim}
##  [1] 0.0000000 0.3010300 0.4771213 0.6020600 0.6989700 0.7781513 0.8450980
##  [8] 0.9030900 0.9542425 1.0000000
\end{verbatim}

\begin{Shaded}
\begin{Highlighting}[]
\FunctionTok{pBendford}\NormalTok{(}\DecValTok{0}\NormalTok{)}
\end{Highlighting}
\end{Shaded}

\begin{verbatim}
## [1] 0
\end{verbatim}

\begin{Shaded}
\begin{Highlighting}[]
\FunctionTok{pBendford}\NormalTok{(}\DecValTok{10}\NormalTok{)}
\end{Highlighting}
\end{Shaded}

\begin{verbatim}
## [1] 1
\end{verbatim}

Así tenemos que

\[
F_X(x)=P(X\leq x)=\left\{
\begin{array}{ll}
0 & \mbox{si } x<1\\
0.30103 & \mbox{si } 1\leq x < 2\\
0.4771213 & \mbox{si } 2\leq x < 3\\
0.60206 & \mbox{si } 3\leq x < 4\\
0.69897 & \mbox{si } 4\leq x < 5\\
0.7781513 & \mbox{si } 5\leq x < 6\\
0.845098 & \mbox{si } 6\leq x < 7\\
0.90309 & \mbox{si } 7\leq x < 8\\
0.9542425 & \mbox{si } 8\leq x < 9\\
1  & \mbox{si } 9\leq x
\end{array}
\right.
\] c) EL dígito más frecuente es el 1

\begin{Shaded}
\begin{Highlighting}[]
\FunctionTok{dBendford}\NormalTok{(}\DecValTok{1}\SpecialCharTok{:}\DecValTok{9}\NormalTok{)}
\end{Highlighting}
\end{Shaded}

\begin{verbatim}
## [1] 0.30103000 0.17609126 0.12493874 0.09691001 0.07918125 0.06694679 0.05799195
## [8] 0.05115252 0.04575749
\end{verbatim}

\begin{Shaded}
\begin{Highlighting}[]
\FunctionTok{max}\NormalTok{(}\FunctionTok{dBendford}\NormalTok{(}\DecValTok{1}\SpecialCharTok{:}\DecValTok{9}\NormalTok{))}
\end{Highlighting}
\end{Shaded}

\begin{verbatim}
## [1] 0.30103
\end{verbatim}

\begin{Shaded}
\begin{Highlighting}[]
\FunctionTok{which.max}\NormalTok{(}\FunctionTok{dBendford}\NormalTok{(}\DecValTok{1}\SpecialCharTok{:}\DecValTok{9}\NormalTok{))}
\end{Highlighting}
\end{Shaded}

\begin{verbatim}
## [1] 1
\end{verbatim}

\begin{enumerate}
\def\labelenumi{\alph{enumi})}
\setcounter{enumi}{3}
\item
  Ya lo hemos hecho\ldots.
\item
  Dibujemos
\end{enumerate}

\begin{Shaded}
\begin{Highlighting}[]
\FunctionTok{par}\NormalTok{(}\AttributeTok{mfrow=}\FunctionTok{c}\NormalTok{(}\DecValTok{1}\NormalTok{,}\DecValTok{2}\NormalTok{))}
\NormalTok{aux}\OtherTok{=}\FunctionTok{rep}\NormalTok{(}\DecValTok{0}\NormalTok{,}\DecValTok{18}\NormalTok{)}
\NormalTok{aux[}\FunctionTok{seq}\NormalTok{(}\DecValTok{2}\NormalTok{,}\DecValTok{18}\NormalTok{,}\DecValTok{2}\NormalTok{)]}\OtherTok{=}\FunctionTok{dBendford}\NormalTok{(}\FunctionTok{c}\NormalTok{(}\DecValTok{1}\SpecialCharTok{:}\DecValTok{9}\NormalTok{))}
\NormalTok{x}\OtherTok{=}\FunctionTok{c}\NormalTok{(}\DecValTok{1}\SpecialCharTok{:}\DecValTok{9}\NormalTok{)}
\FunctionTok{plot}\NormalTok{(x,}\AttributeTok{y=}\FunctionTok{dBendford}\NormalTok{(}\FunctionTok{c}\NormalTok{(}\DecValTok{1}\SpecialCharTok{:}\DecValTok{9}\NormalTok{)),}
  \AttributeTok{ylim=}\FunctionTok{c}\NormalTok{(}\DecValTok{0}\NormalTok{,}\DecValTok{1}\NormalTok{),}\AttributeTok{xlim=}\FunctionTok{c}\NormalTok{(}\DecValTok{0}\NormalTok{,}\DecValTok{10}\NormalTok{),}\AttributeTok{xlab=}\StringTok{"x"}\NormalTok{,}
  \AttributeTok{main=}\StringTok{"Función de probabilidad }\SpecialCharTok{\textbackslash{}n}\StringTok{ Bendford"}\NormalTok{)}
\FunctionTok{lines}\NormalTok{(}\AttributeTok{x=}\FunctionTok{rep}\NormalTok{(}\DecValTok{1}\SpecialCharTok{:}\DecValTok{9}\NormalTok{,}\AttributeTok{each=}\DecValTok{2}\NormalTok{),}\AttributeTok{y=}\NormalTok{aux, }\AttributeTok{type =} \StringTok{"h"}\NormalTok{, }\AttributeTok{lty =} \DecValTok{2}\NormalTok{,}\AttributeTok{col=}\StringTok{"blue"}\NormalTok{)}
\FunctionTok{curve}\NormalTok{(}\FunctionTok{pBendford}\NormalTok{(x), }\AttributeTok{xlim=}\FunctionTok{c}\NormalTok{(}\SpecialCharTok{{-}}\DecValTok{1}\NormalTok{,}\DecValTok{11}\NormalTok{),}\AttributeTok{col=}\StringTok{"blue"}\NormalTok{, }\AttributeTok{main=}\StringTok{"Función de distribución}\SpecialCharTok{\textbackslash{}n}\StringTok{ Bendford"}\NormalTok{)}
\end{Highlighting}
\end{Shaded}

\includegraphics{Tema-3---Notables_mas_notables_SOLUCIONES_files/figure-latex/unnamed-chunk-5-1.pdf}

\begin{Shaded}
\begin{Highlighting}[]
\FunctionTok{par}\NormalTok{(}\AttributeTok{mfrow=}\FunctionTok{c}\NormalTok{(}\DecValTok{1}\NormalTok{,}\DecValTok{1}\NormalTok{))}
\end{Highlighting}
\end{Shaded}

\hypertarget{distribuciuxf3n-de-pareto-power-law}{%
\subsection{\texorpdfstring{\textbf{Distribución de Pareto}
(\textbf{Power
law})}{Distribución de Pareto (Power law)}}\label{distribuciuxf3n-de-pareto-power-law}}

Es una distribución que aparece en muchos ámbitos. Consideremos el
económico. Supongamos que en un gran país consideramos la población
activa económicamente; desde el más humilde becario al directivo más
adinerado.

Escogemos un individuo al azar de esta población y observamos la
variable \(X=\) sus ingresos en euros (digamos que anuales).

Un modelo razonable es el que supone que:

\begin{itemize}
\tightlist
\item
  Hay un ingreso mínimo \(x_m>0\).
\item
  La probabilidad de un ingreso mayor que \(x\) decrece de forma
  inversamente proporcional al ingreso \(x\), es decir proporcional a
  \(\left(\frac{x_m}{x}\right)^{\gamma}\) para algún número real
  \(\gamma >1.\)
\end{itemize}

Más formalmente. dado \(x>x_m\)

\[P(X>x)=k\cdot \left(\frac{x_m}{x}\right)^{\gamma}.\] Luego su función
de distribución es

\[
F_X(X)=P(X\leq x)=\left\{
\begin{array}{ll}
1-P(X > x)= 1- k\cdot \left(\frac{x_m}{x}\right)^{\gamma} & \mbox{ si } x>x_m\\
0 & \mbox{ si } x\leq x_m
\end{array}
\right.
\] Se pide

\begin{itemize}
\item
  \begin{enumerate}
  \def\labelenumi{\alph{enumi})}
  \tightlist
  \item
    Calcular en función de \(k\) y \(\gamma\) la densidad de la variable
    \(X\).
  \end{enumerate}
\item
  \begin{enumerate}
  \def\labelenumi{\alph{enumi})}
  \setcounter{enumi}{1}
  \tightlist
  \item
    Para \(\gamma>1\) calcular \(E(X)\) y \(Var(X)\) y su desviación
    típica.
  \end{enumerate}
\item
  \begin{enumerate}
  \def\labelenumi{\alph{enumi})}
  \setcounter{enumi}{2}
  \tightlist
  \item
    ¿Qué sucede con \(E(X)\) si \(0<\gamma<1\).
  \end{enumerate}
\item
  \begin{enumerate}
  \def\labelenumi{\alph{enumi})}
  \setcounter{enumi}{3}
  \tightlist
  \item
    ¿Cómo se calcula está
    \href{https://www.rdocumentation.org/packages/EnvStats/versions/2.3.1/topics/Pareto}{distribución
    con R}
    \href{https://docs.scipy.org/doc/scipy/reference/generated/scipy.stats.Pareto.html}{y
    con python}?
  \end{enumerate}
\item
  \begin{enumerate}
  \def\labelenumi{\alph{enumi})}
  \setcounter{enumi}{4}
  \tightlist
  \item
    Dibujar las gráficas de su densidad y distribución para \(\gamma=3\)
    y \(\gamma=5\).
  \end{enumerate}
\item
  \begin{enumerate}
  \def\labelenumi{\alph{enumi})}
  \setcounter{enumi}{5}
  \tightlist
  \item
    Explorar por internet (wikipedia) cómo es la distribución
    \textbf{power law} y qué relación tiene el concepto de \emph{scale
    free} con los resultados del apartado c).
  \end{enumerate}
\end{itemize}

\hypertarget{soluciuxf3n-1}{%
\subsubsection{Solución}\label{soluciuxf3n-1}}

\begin{enumerate}
\def\labelenumi{\alph{enumi})}
\tightlist
\item
  La densidad será la derivada de la distribución \(F_X\) respecto de
  \(x\) , si \(x\geq x_m>0\)
\end{enumerate}

\begin{eqnarray*}
f_X(x)=(F_X(x))'&=& \left(1- k\cdot \left(\frac{x_m}{x}\right)^{\gamma}\right)'=
\left(1- k\cdot x_m^{\gamma} \cdot  x^{-\gamma} \right)'= \left(1- k\cdot x_m^{\gamma} \cdot  x^{-\gamma} \right)'\\
&=& -\gamma\cdot(-k\cdot x_m^{\gamma}) \cdot  x^{-\gamma-1}=\gamma\cdot k\cdot x_m^{\gamma} \cdot  x^{-\gamma-1}
\end{eqnarray*}

Si tenemos \(x<x_m\) entonces \(f_X(x)=0\) en resumen

\[
f_X(x)=\left\{
\begin{array}{ll}
\gamma \cdot x_m^{\gamma} \cdot  x^{-\gamma-1} &\mbox{ si } x\geq x_m\\
0 &\mbox{ si } x< x_m\\
\end{array}
\right.
\] Notemos que \(\gamma\) es un parámetro pero \(k\) es na constante a
determinar pues la densidad debe integrar 1 en el dominio
\(D_X=[x_m,+\infty)\)

\begin{eqnarray*}
\int_{x_m}^{+\infty} f_X(x) dx &=&\int_{x_m}^{+\infty} \gamma \cdot k\cdot x_m^{\gamma} \cdot  x^{-\gamma-1}\cdot dx=
\left[- k\cdot x_m^{\gamma} \cdot  x^{-\gamma}\right]_{x=x_m}^{+\infty}=
\lim_{x\to \infty} \left[- k\cdot x_m^{\gamma} \cdot  x^{-\gamma}\right] - \left(- k\cdot x_m^{\gamma} \cdot  x_m^{-\gamma}\right)\\
&=& 0 + k\cdot x_m^{\gamma} \cdot  {x_m}^{-\gamma}=k.
\end{eqnarray*}

Notemos que el límite es \(0\) pues \(\gamma>0\) y \(x_m>0\). Luego
\(k=1\) y la función de densidad y la de distribución se puede escribir
como damos dos versiones

\[
f_X(x)=\left\{
\begin{array}{ll}
\gamma\cdot x_m^{\gamma} \cdot  x^{-(\gamma+1)} &\mbox{ si } x\geq x_m\\
0 &\mbox{ si } x< x_m\\
\end{array}
\right.=
\left\{
\begin{array}{ll}
\frac{\gamma\cdot x_m^{\gamma}}{x^{(\gamma+1)}} &\mbox{ si } x\geq x_m\\
0 &\mbox{ si } x< x_m\\
\end{array}
\right.
\]

\[
f_X(x)=\left\{
\begin{array}{ll}
\gamma\cdot x_m^{\gamma} \cdot  x^{-(\gamma+1)} &\mbox{ si } x\geq x_m\\
0 &\mbox{ si } x< x_m\\
\end{array}
\right.=
\left\{
\begin{array}{ll}
\frac{\gamma\cdot x_m^{\gamma}}{x^{(\gamma+1)}} &\mbox{ si } x\geq x_m\\
0 &\mbox{ si } x< x_m\\
\end{array}
\right.
\]

y la distribución

\[
F_X(X)=
\left\{
\begin{array}{ll}
1-  x_m^\gamma \cdot x^{-\gamma} & \mbox{ si } x>x_m\\
0 & \mbox{ si } x\leq x_m
\end{array}
\right.
=\left\{
\begin{array}{ll}
1-  \left(\frac{x_m}{x}\right)^{\gamma} & \mbox{ si } x>x_m\\
0 & \mbox{ si } x\leq x_m
\end{array}
\right.
\]

\begin{enumerate}
\def\labelenumi{\alph{enumi})}
\setcounter{enumi}{2}
\tightlist
\item
\end{enumerate}

Calculemos su esperanza

\begin{eqnarray*}
E(X)&=&\int_{x_m}^{+\infty} x\cdot f_X(x)\cdot dx=
\int_{x_m}^{+\infty} x\cdot \gamma\cdot x_m^{\gamma} \cdot  x^{-\gamma-1}\cdot dx=
\int_{x_m}^{+\infty} \gamma\cdot x_m^{\gamma} \cdot  x^{-\gamma}\cdot dx=
\left[\frac{\gamma}{-\gamma+1}\cdot x_m^{\gamma} \cdot  x^{-\gamma+1}\right]_{x=x_m}^{+\infty}
\\
&=& \lim_{x\to \infty} \left[\frac{\gamma}{-\gamma+1}\cdot x_m^{\gamma} \cdot  x^{-\gamma+1}\right] - \left(\frac{\gamma}{-\gamma+1} \cdot x_m^{\gamma} \cdot  x_m^{-\gamma+1}\right)=
\lim_{x\to \infty} \left[\frac{\gamma}{-\gamma+1}\cdot x_m^{\gamma} \cdot  x^{-\gamma+1}\right] + \frac{\gamma\cdot x_m}{\gamma-1}  
\end{eqnarray*}

Ahora tenemos dos casos para el límite que \(0<\gamma\leq 1\) o que
\(\gamma>1\), es decir que \(-\gamma+1\) sea negativo o positivo,
entonces

\[
\lim_{x\to \infty} 
\left[
\frac{\gamma}{-\gamma+1}\cdot x_m^{\gamma} \cdot  x^{-\gamma+1}
\right]=
\left\{
\begin{array}{ll}
+\infty & \mbox{ diverge si }  0<\gamma\leq 1 \\
\frac{\gamma\cdot x_m}{\gamma-1}   & \mbox{ converge si } \gamma > 1\\
\end{array}
\right.
\]

Así que \textbf{no siempre existe} \(E(X)\), si en una distribución
Pareto \(\gamma\leq 1\) su media diverge se dice entonces que es una
distribución \textbf{de escala libre}, en inglés \textbf{scale free} en
el sentido de que carece de media.

\begin{enumerate}
\def\labelenumi{\alph{enumi})}
\setcounter{enumi}{4}
\tightlist
\item
  Podemos programar pero ya lo han hecho en el paquete
  \emph{Environmental Statistics} (\texttt{EnvStats}) y el \emph{Extra
  Distributions} (\texttt{extraDistr}) utilizaremos el segundo paquete
  en el que las funciones están implementadas en C++) instalarlo si no
  lo tenéis.
\end{enumerate}

\begin{Shaded}
\begin{Highlighting}[]
\FunctionTok{par}\NormalTok{(}\AttributeTok{mfrow=}\FunctionTok{c}\NormalTok{(}\DecValTok{1}\NormalTok{,}\DecValTok{2}\NormalTok{))}\CommentTok{\# el parámetro gamma es a y el parámetro m es b}
\FunctionTok{curve}\NormalTok{(extraDistr}\SpecialCharTok{::}\FunctionTok{dpareto}\NormalTok{(x,}\AttributeTok{a=}\DecValTok{1}\NormalTok{,}\AttributeTok{b=}\DecValTok{1}\NormalTok{),}\AttributeTok{xlim=}\FunctionTok{c}\NormalTok{(}\DecValTok{1}\NormalTok{,}\DecValTok{50}\NormalTok{),}
      \AttributeTok{ylim=}\FunctionTok{c}\NormalTok{(}\DecValTok{0}\NormalTok{,}\FloatTok{0.04}\NormalTok{),}\AttributeTok{lty=}\DecValTok{1}\NormalTok{,}\AttributeTok{main=}\StringTok{"Densidad  Pareto."}
\NormalTok{      ,}\AttributeTok{ylab=}\StringTok{"Densidad Pareto"}\NormalTok{)}
\FunctionTok{curve}\NormalTok{(extraDistr}\SpecialCharTok{::}\FunctionTok{dpareto}\NormalTok{(x,}\AttributeTok{a=}\DecValTok{3}\NormalTok{,}\AttributeTok{b=}\DecValTok{4}\NormalTok{),}
      \AttributeTok{add=}\ConstantTok{TRUE}\NormalTok{,}\AttributeTok{col=}\StringTok{"red"}\NormalTok{,}\AttributeTok{lty=}\DecValTok{2}\NormalTok{)}
\FunctionTok{curve}\NormalTok{(extraDistr}\SpecialCharTok{::}\FunctionTok{dpareto}\NormalTok{(x,}\AttributeTok{a=}\DecValTok{4}\NormalTok{,}\AttributeTok{b=}\DecValTok{10}\NormalTok{),}
      \AttributeTok{add=}\ConstantTok{TRUE}\NormalTok{,}\AttributeTok{col=}\StringTok{"green"}\NormalTok{,}\AttributeTok{lty=}\DecValTok{3}\NormalTok{)}
\FunctionTok{legend}\NormalTok{(}\StringTok{"topright"}\NormalTok{,}\AttributeTok{pch=}\DecValTok{21}\NormalTok{,}
       \AttributeTok{legend=}\FunctionTok{c}\NormalTok{(}\StringTok{"gamma=1 xm=1"}\NormalTok{,}\StringTok{"gamma=3 xm=4"}\NormalTok{,}\StringTok{"gamma=4 xm=10"}\NormalTok{),}
       \AttributeTok{col=}\FunctionTok{c}\NormalTok{(}\StringTok{"black"}\NormalTok{,}\StringTok{"red"}\NormalTok{,}\StringTok{"green"}\NormalTok{),}\AttributeTok{lty=}\FunctionTok{c}\NormalTok{(}\DecValTok{1}\NormalTok{,}\DecValTok{2}\NormalTok{,}\DecValTok{3}\NormalTok{),}\AttributeTok{cex=}\FloatTok{0.5}\NormalTok{)}
\FunctionTok{curve}\NormalTok{(extraDistr}\SpecialCharTok{::}\FunctionTok{ppareto}\NormalTok{(x,}\AttributeTok{a=}\DecValTok{1}\NormalTok{,}\AttributeTok{b=}\DecValTok{1}\NormalTok{),}
      \AttributeTok{xlim=}\FunctionTok{c}\NormalTok{(}\DecValTok{1}\NormalTok{,}\DecValTok{50}\NormalTok{),}\AttributeTok{ylim=}\FunctionTok{c}\NormalTok{(}\DecValTok{0}\NormalTok{,}\DecValTok{1}\NormalTok{),}\AttributeTok{lty=}\DecValTok{1}\NormalTok{,}\AttributeTok{main=}\StringTok{"Distribución  Pareto."}\NormalTok{,}
      \AttributeTok{ylab=}\StringTok{"Distribución acumulada Pareto"}\NormalTok{)}
\FunctionTok{curve}\NormalTok{(extraDistr}\SpecialCharTok{::}\FunctionTok{ppareto}\NormalTok{(x,}\AttributeTok{a=}\DecValTok{3}\NormalTok{,}\AttributeTok{b=}\DecValTok{4}\NormalTok{),}
      \AttributeTok{add=}\ConstantTok{TRUE}\NormalTok{,}\AttributeTok{col=}\StringTok{"red"}\NormalTok{,}\AttributeTok{lty=}\DecValTok{2}\NormalTok{)}
\FunctionTok{curve}\NormalTok{(extraDistr}\SpecialCharTok{::}\FunctionTok{ppareto}\NormalTok{(x,}\AttributeTok{a=}\DecValTok{4}\NormalTok{,}\AttributeTok{b=}\DecValTok{10}\NormalTok{),}
      \AttributeTok{add=}\ConstantTok{TRUE}\NormalTok{,}\AttributeTok{col=}\StringTok{"green"}\NormalTok{,}\AttributeTok{lty=}\DecValTok{3}\NormalTok{)}
\end{Highlighting}
\end{Shaded}

\includegraphics{Tema-3---Notables_mas_notables_SOLUCIONES_files/figure-latex/unnamed-chunk-6-1.pdf}

\begin{Shaded}
\begin{Highlighting}[]
\FunctionTok{par}\NormalTok{(}\AttributeTok{mfrow=}\FunctionTok{c}\NormalTok{(}\DecValTok{1}\NormalTok{,}\DecValTok{1}\NormalTok{))}
\end{Highlighting}
\end{Shaded}

\begin{enumerate}
\def\labelenumi{\alph{enumi})}
\setcounter{enumi}{5}
\tightlist
\item
  Buscad los enlaces del la wikipedia. Tenéis que buscar la \emph{Power
  law} y la \emph{Zipf's law}. Ambas distribuciones son famosas aparecen
  en la distribución de contactos en una ley social, en la longitud de
  un mensaje en un foro y en otros aspectos empíricos muy interesantes.
  Si hay ocasión y el curso es un éxito ampliaremos estas
  distribuciones.
\end{enumerate}

\hypertarget{distribuciuxf3n-de-gumbel-teoruxeda-del-valor-extremo.}{%
\subsection{\texorpdfstring{\textbf{Distribución de Gumbel (teoría del
valor
extremo)}.}{Distribución de Gumbel (teoría del valor extremo).}}\label{distribuciuxf3n-de-gumbel-teoruxeda-del-valor-extremo.}}

La distribución de Gumbel aparece en variables que miden lo que se llama
un valor extremo: precipitación máxima de lluvia, tiempo máximo
transcurrido entre dos terremotos, o en métodos de \emph{machine
learning} el máximo de las puntuaciones de una algoritmo; por ejemplo
comparar pares de objetos (fotos, proteínas, etc.).

Una variable aleatoria sigue una ley de distribución Gumbel (de TIPO I)
si su distribución es:

\[
F_X(x)=\left\{
\begin{array}{ll}
  e^{-e^{-\frac{x-\mu}{\beta}}} & \mbox{si} x\geq 0\\
 0 & \mbox{si} x< 0\\
\end{array}
\right.
\]

Para \(\mu\) y \(\beta>0\) parámetros reales. Llamaremos distribución
Gumbel estándar a la que tiene por parámetros \(\mu=0\) y \(\beta=1.\)

\begin{itemize}
\item
  \begin{enumerate}
  \def\labelenumi{\alph{enumi})}
  \tightlist
  \item
    Si \(X\) es una Gumbel estándar calcular su función de densidad y
    dibujar su gráfica.
  \end{enumerate}
\item
  \begin{enumerate}
  \def\labelenumi{\alph{enumi})}
  \setcounter{enumi}{1}
  \tightlist
  \item
    Consideremos la función \(F(x)=e^{-e^{-x}}\) para \(x\geq 0\) y que
    vale cero en el resto de casos. Comprobar que es la función de
    distribución \(P(X\leq x)\) de una v.a. Gumbel estándar.
  \end{enumerate}
\item
  \begin{enumerate}
  \def\labelenumi{\alph{enumi})}
  \setcounter{enumi}{2}
  \tightlist
  \item
    Buscad un paquete de R que implemente la distribución Gumbel.
    Aseguraros de que es la (Gumbel Tipo I). Dejando fijo el parámetro
    \(\beta=1\) dibujar la densidad Gumbel para varios valores de
    \(\mu\) y explicad en que afecta a la gráfica el cambio de \(\mu\).
  \end{enumerate}
\item
  \begin{enumerate}
  \def\labelenumi{\alph{enumi})}
  \setcounter{enumi}{3}
  \tightlist
  \item
    Dejando fijo el parámetro \(\mu\) dibujad la densidad Gumbel para
    varios valores de \(\beta>0\) y explicar en qué afecta a la gráfica
    el cambio de este parámetro.
  \end{enumerate}
\item
  \begin{enumerate}
  \def\labelenumi{\alph{enumi})}
  \setcounter{enumi}{4}
  \tightlist
  \item
    Buscad cuales son las fórmulas de la esperanza y varianza de una
    distribución Gumbel en función de \(\alpha\) y \(\beta\).
  \end{enumerate}
\item
  \begin{enumerate}
  \def\labelenumi{\alph{enumi})}
  \setcounter{enumi}{5}
  \tightlist
  \item
    Repetid los apartados c) y d) con python. Con python se puede pedir
    con la correspondiente función la esperanza y varianza de esta
    distribución, comprobar con esta función para algunos valores las
    fórmulas de la esperanza y la varianza del apartado e).
  \end{enumerate}
\end{itemize}

\hypertarget{soluciuxf3n-2}{%
\subsubsection{Solución}\label{soluciuxf3n-2}}

\begin{enumerate}
\def\labelenumi{\alph{enumi})}
\tightlist
\item
  La Gumbel estándar tiene por distribución
\end{enumerate}

\[
F_X(x)=\left\{
\begin{array}{ll}
  e^{-e^{-x}} & \mbox{ si } x\geq 0\\
 0 & \mbox{ si } x< 0\\
\end{array}
\right.
\]

Entonces si \(x>0\)

\[f_X(X)=(F_X(x))'=\left(e^{-e^{-x}}\right)'=e^{-e^{-x}}\cdot e^{-x} \]

Luego

\[f_X(x)=\left\{
\begin{array}{ll}
  e^{-e^{-x}}\cdot e^{-x} & \mbox{ si } x\geq 0\\
 0 & \mbox{ si } x< 0\\
\end{array}
\right.
\]

\begin{Shaded}
\begin{Highlighting}[]
\NormalTok{dgumbel\_standar}\OtherTok{=}\ControlFlowTok{function}\NormalTok{(x) \{}
  \FunctionTok{sapply}\NormalTok{(x,}
         \AttributeTok{FUN=}\ControlFlowTok{function}\NormalTok{(x) \{}
           \ControlFlowTok{if}\NormalTok{(x}\SpecialCharTok{\textgreater{}}\DecValTok{0}\NormalTok{) \{}\FunctionTok{return}\NormalTok{(}\FunctionTok{exp}\NormalTok{(}\SpecialCharTok{{-}}\FunctionTok{exp}\NormalTok{(}\SpecialCharTok{{-}}\NormalTok{x))}\SpecialCharTok{*}\FunctionTok{exp}\NormalTok{(}\SpecialCharTok{{-}}\NormalTok{x))\} }\ControlFlowTok{else}\NormalTok{ \{}\FunctionTok{return}\NormalTok{(}\DecValTok{0}\NormalTok{)\}}
\NormalTok{         \}}
\NormalTok{  )}
\NormalTok{  \}}

\FunctionTok{curve}\NormalTok{(}\FunctionTok{dgumbel\_standar}\NormalTok{(x),}\AttributeTok{col=}\StringTok{"blue"}\NormalTok{,}\AttributeTok{main=}\StringTok{"Densidad  Gummbel estándar"}\NormalTok{,}\AttributeTok{xlim=}\FunctionTok{c}\NormalTok{(}\DecValTok{0}\NormalTok{,}\DecValTok{10}\NormalTok{))}
\end{Highlighting}
\end{Shaded}

\includegraphics{Tema-3---Notables_mas_notables_SOLUCIONES_files/figure-latex/plot_gumbel_standar-1.pdf}

\begin{enumerate}
\def\labelenumi{\alph{enumi})}
\setcounter{enumi}{1}
\tightlist
\item
  Consideremos la función \(F(x)=e^{-e^{-x}}\) para \(x\geq 0\) y que
  vale cero en el resto de casos. Comprobar que es la función de
  distribución \(P(X\leq x)\) de una v.a. Gumbel estándar.
\end{enumerate}

Efectivamente Es suficiente sustituir en la fórmula original.

\begin{enumerate}
\def\labelenumi{\alph{enumi})}
\setcounter{enumi}{2}
\tightlist
\item
  Buscad un paquete de R que implemente la distribución Gumbel.
  Aseguraros de que es la (Gumbel Tipo I). Dejando fijo el parámetro
  \(\beta=1\) dibujar la densidad Gumbel para varios valores de \(\mu\)
  y explicad en que afecta a la gráfica el cambio de \(\mu\).
\end{enumerate}

Un paquete que implementa la Gumbel es \texttt{extraDistr} el parámetro
\texttt{mu} es \(\mu\) mientras que \(\beta\) es el parámetro
\texttt{sigma}.

\begin{Shaded}
\begin{Highlighting}[]
\CommentTok{\# el parámetro mu es mu y el parámetro beta es sigma}
\FunctionTok{curve}\NormalTok{(extraDistr}\SpecialCharTok{::}\FunctionTok{dgumbel}\NormalTok{(x,}\AttributeTok{mu=}\DecValTok{1}\NormalTok{,}\AttributeTok{sigma=}\DecValTok{1}\NormalTok{),}\AttributeTok{xlim=}\FunctionTok{c}\NormalTok{(}\DecValTok{0}\NormalTok{,}\DecValTok{10}\NormalTok{),}
      \AttributeTok{ylim=}\FunctionTok{c}\NormalTok{(}\DecValTok{0}\NormalTok{,}\FloatTok{0.38}\NormalTok{),}\AttributeTok{lty=}\DecValTok{1}\NormalTok{,}\AttributeTok{main=}\StringTok{"Densidad  Gumbel}\SpecialCharTok{\textbackslash{}n}\StringTok{ mu=1, 2 y 4  y beta 1"}\NormalTok{)}
\FunctionTok{curve}\NormalTok{(extraDistr}\SpecialCharTok{::}\FunctionTok{dgumbel}\NormalTok{(x,}\AttributeTok{mu=}\DecValTok{2}\NormalTok{,}\AttributeTok{sigma=}\DecValTok{1}\NormalTok{),}
      \AttributeTok{add=}\ConstantTok{TRUE}\NormalTok{,}\AttributeTok{col=}\StringTok{"red"}\NormalTok{,}\AttributeTok{lty=}\DecValTok{2}\NormalTok{)}
\FunctionTok{curve}\NormalTok{(extraDistr}\SpecialCharTok{::}\FunctionTok{dgumbel}\NormalTok{(x,}\AttributeTok{mu=}\DecValTok{4}\NormalTok{,}\AttributeTok{sigma=}\DecValTok{1}\NormalTok{),}
      \AttributeTok{add=}\ConstantTok{TRUE}\NormalTok{,}\AttributeTok{col=}\StringTok{"green"}\NormalTok{,}\AttributeTok{lty=}\DecValTok{3}\NormalTok{)}
\FunctionTok{legend}\NormalTok{(}\StringTok{"topright"}\NormalTok{,}\AttributeTok{pch=}\DecValTok{21}\NormalTok{,}
       \AttributeTok{legend=}\FunctionTok{c}\NormalTok{(}\StringTok{"mu=1, beta=1"}\NormalTok{,}\StringTok{"mu=2, beta=1"}\NormalTok{,}\StringTok{"mu=4, beta=1"}\NormalTok{),}
       \AttributeTok{col=}\FunctionTok{c}\NormalTok{(}\StringTok{"black"}\NormalTok{,}\StringTok{"red"}\NormalTok{,}\StringTok{"green"}\NormalTok{),}\AttributeTok{lty=}\FunctionTok{c}\NormalTok{(}\DecValTok{1}\NormalTok{,}\DecValTok{2}\NormalTok{,}\DecValTok{3}\NormalTok{),}\AttributeTok{cex=}\FloatTok{0.5}\NormalTok{)}
\end{Highlighting}
\end{Shaded}

\includegraphics{Tema-3---Notables_mas_notables_SOLUCIONES_files/figure-latex/plot_gumbel_taller_e-1.pdf}

\begin{enumerate}
\def\labelenumi{\alph{enumi})}
\setcounter{enumi}{3}
\tightlist
\item
  Dejando fijo el parámetro \(\mu\) dibujad la densidad Gumbel para
  varios valores de \(\beta>0\) y explicar en que afecta a la gráfica el
  cambio de este parámetro.
\end{enumerate}

\begin{Shaded}
\begin{Highlighting}[]
\CommentTok{\# el paramento mu es mu y el parámetro beta es sigma}
\FunctionTok{curve}\NormalTok{(extraDistr}\SpecialCharTok{::}\FunctionTok{dgumbel}\NormalTok{(x,}\AttributeTok{mu=}\DecValTok{1}\NormalTok{,}\AttributeTok{sigma=}\DecValTok{1}\NormalTok{),}\AttributeTok{xlim=}\FunctionTok{c}\NormalTok{(}\DecValTok{0}\NormalTok{,}\DecValTok{15}\NormalTok{),}
      \AttributeTok{ylim=}\FunctionTok{c}\NormalTok{(}\DecValTok{0}\NormalTok{,}\FloatTok{0.38}\NormalTok{),}\AttributeTok{lty=}\DecValTok{1}\NormalTok{,}\AttributeTok{main=}\StringTok{"Densidad  Gumbel}\SpecialCharTok{\textbackslash{}n}\StringTok{ mu=1 y beta variando 1,5, y 10"}\NormalTok{)}
\FunctionTok{curve}\NormalTok{(extraDistr}\SpecialCharTok{::}\FunctionTok{dgumbel}\NormalTok{(x,}\AttributeTok{mu=}\DecValTok{1}\NormalTok{,}\AttributeTok{sigma=}\DecValTok{2}\NormalTok{),}
      \AttributeTok{add=}\ConstantTok{TRUE}\NormalTok{,}\AttributeTok{col=}\StringTok{"red"}\NormalTok{,}\AttributeTok{lty=}\DecValTok{2}\NormalTok{)}
\FunctionTok{curve}\NormalTok{(extraDistr}\SpecialCharTok{::}\FunctionTok{dgumbel}\NormalTok{(x,}\AttributeTok{mu=}\DecValTok{1}\NormalTok{,}\AttributeTok{sigma=}\DecValTok{4}\NormalTok{),}
      \AttributeTok{add=}\ConstantTok{TRUE}\NormalTok{,}\AttributeTok{col=}\StringTok{"green"}\NormalTok{,}\AttributeTok{lty=}\DecValTok{3}\NormalTok{)}
\FunctionTok{legend}\NormalTok{(}\StringTok{"topright"}\NormalTok{,}\AttributeTok{pch=}\DecValTok{21}\NormalTok{,}
       \AttributeTok{legend=}\FunctionTok{c}\NormalTok{(}\StringTok{"mu=1, beta=1"}\NormalTok{,}\StringTok{"mu=1, beta=2"}\NormalTok{,}\StringTok{"mu=1, beta=4"}\NormalTok{),}
       \AttributeTok{col=}\FunctionTok{c}\NormalTok{(}\StringTok{"black"}\NormalTok{,}\StringTok{"red"}\NormalTok{,}\StringTok{"green"}\NormalTok{),}\AttributeTok{lty=}\FunctionTok{c}\NormalTok{(}\DecValTok{1}\NormalTok{,}\DecValTok{2}\NormalTok{,}\DecValTok{3}\NormalTok{),}\AttributeTok{cex=}\FloatTok{0.5}\NormalTok{)}
\end{Highlighting}
\end{Shaded}

\includegraphics{Tema-3---Notables_mas_notables_SOLUCIONES_files/figure-latex/plot_gumbel_taller_d-1.pdf}

\begin{enumerate}
\def\labelenumi{\alph{enumi})}
\setcounter{enumi}{4}
\item
  Si \(X\) sigue un ley Gumbel de parámetros \(\mu\) y \(\beta\)
  entonces \(E(X)=\mu+\beta\cdot\gamma\) donde gamma es el número de
  euler \(\gamma=0.577215664\ldots\), y
  \(Var(X)=\frac{\pi^2}{6}\cdot \beta^2\)
\item
  Repetid los apartados c) y d) con python. Con python se puede pedir
  con la correspondiente función la esperanza y varianza de esta
  distribución, comprobar con esta función para algunos valores las
  fórmulas de la esperanza y la varianza del apartado e).
\end{enumerate}

\begin{Shaded}
\begin{Highlighting}[]
\ImportTok{import}\NormalTok{ numpy }\ImportTok{as}\NormalTok{ np }
\ImportTok{from}\NormalTok{ scipy.stats }\ImportTok{import}\NormalTok{ gumbel\_r}
\NormalTok{mu, beta }\OperatorTok{=} \DecValTok{0}\NormalTok{, }\FloatTok{0.1} \CommentTok{\# location and scale}
\NormalTok{x }\OperatorTok{=}\NormalTok{ np.linspace(}\DecValTok{0}\NormalTok{, }\DecValTok{2} \OperatorTok{*}\NormalTok{ np.pi, }\DecValTok{30}\NormalTok{)}
\ImportTok{import}\NormalTok{ matplotlib.pyplot }\ImportTok{as}\NormalTok{ plt}
\CommentTok{\#count, bins, ignored = plt.hist(s, 30, normed=True)}
\NormalTok{plt.plot(x,gumbel\_r.pdf(x, loc}\OperatorTok{=}\DecValTok{1}\NormalTok{, scale}\OperatorTok{=}\DecValTok{1}\NormalTok{),linewidth}\OperatorTok{=}\DecValTok{2}\NormalTok{, color}\OperatorTok{=}\StringTok{\textquotesingle{}black\textquotesingle{}}\NormalTok{)}
\NormalTok{plt.plot(x,gumbel\_r.pdf(x, loc}\OperatorTok{=}\DecValTok{2}\NormalTok{, scale}\OperatorTok{=}\DecValTok{1}\NormalTok{),linewidth}\OperatorTok{=}\DecValTok{2}\NormalTok{, color}\OperatorTok{=}\StringTok{\textquotesingle{}green\textquotesingle{}}\NormalTok{)}
\NormalTok{plt.plot(x,gumbel\_r.pdf(x, loc}\OperatorTok{=}\DecValTok{4}\NormalTok{, scale}\OperatorTok{=}\DecValTok{1}\NormalTok{),linewidth}\OperatorTok{=}\DecValTok{2}\NormalTok{, color}\OperatorTok{=}\StringTok{\textquotesingle{}red\textquotesingle{}}\NormalTok{)}
\NormalTok{plt.show()}
\end{Highlighting}
\end{Shaded}

\includegraphics{Tema-3---Notables_mas_notables_SOLUCIONES_files/figure-latex/unnamed-chunk-7-1.pdf}

\begin{Shaded}
\begin{Highlighting}[]
\ImportTok{import}\NormalTok{ numpy }\ImportTok{as}\NormalTok{ np }
\ImportTok{from}\NormalTok{ scipy.stats }\ImportTok{import}\NormalTok{ gumbel\_r}
\NormalTok{mu, beta }\OperatorTok{=} \DecValTok{0}\NormalTok{, }\FloatTok{0.1} \CommentTok{\# location and scale}
\NormalTok{x }\OperatorTok{=}\NormalTok{ np.linspace(}\DecValTok{0}\NormalTok{, }\DecValTok{2} \OperatorTok{*}\NormalTok{ np.pi, }\DecValTok{30}\NormalTok{)}
\ImportTok{import}\NormalTok{ matplotlib.pyplot }\ImportTok{as}\NormalTok{ plt}
\CommentTok{\#count, bins, ignored = plt.hist(s, 30, normed=True)}
\NormalTok{plt.plot(x,gumbel\_r.pdf(x, loc}\OperatorTok{=}\DecValTok{1}\NormalTok{, scale}\OperatorTok{=}\DecValTok{1}\NormalTok{),linewidth}\OperatorTok{=}\DecValTok{2}\NormalTok{, color}\OperatorTok{=}\StringTok{\textquotesingle{}black\textquotesingle{}}\NormalTok{)}
\NormalTok{plt.plot(x,gumbel\_r.pdf(x, loc}\OperatorTok{=}\DecValTok{1}\NormalTok{, scale}\OperatorTok{=}\DecValTok{2}\NormalTok{),linewidth}\OperatorTok{=}\DecValTok{2}\NormalTok{, color}\OperatorTok{=}\StringTok{\textquotesingle{}green\textquotesingle{}}\NormalTok{)}
\NormalTok{plt.plot(x,gumbel\_r.pdf(x, loc}\OperatorTok{=}\DecValTok{1}\NormalTok{, scale}\OperatorTok{=}\DecValTok{4}\NormalTok{),linewidth}\OperatorTok{=}\DecValTok{2}\NormalTok{, color}\OperatorTok{=}\StringTok{\textquotesingle{}red\textquotesingle{}}\NormalTok{)}
\NormalTok{plt.show()}
\end{Highlighting}
\end{Shaded}

\includegraphics{Tema-3---Notables_mas_notables_SOLUCIONES_files/figure-latex/unnamed-chunk-8-3.pdf}
Y los estadísticos

\begin{Shaded}
\begin{Highlighting}[]
\ImportTok{from}\NormalTok{ scipy.stats }\ImportTok{import}\NormalTok{ gumbel\_r}
\NormalTok{gumbel\_r.stats(loc}\OperatorTok{=}\DecValTok{0}\NormalTok{, scale}\OperatorTok{=}\DecValTok{1}\NormalTok{, moments}\OperatorTok{=}\StringTok{\textquotesingle{}mv\textquotesingle{}}\NormalTok{)}
\end{Highlighting}
\end{Shaded}

\begin{verbatim}
## (0.5772156649015329, 1.6449340668482264)
\end{verbatim}

\begin{Shaded}
\begin{Highlighting}[]
\BuiltInTok{print}\NormalTok{(}\StringTok{"E(X) = }\SpecialCharTok{\{m\}}\StringTok{"}\NormalTok{.}\BuiltInTok{format}\NormalTok{(m}\OperatorTok{=}\NormalTok{gumbel\_r.stats(loc}\OperatorTok{=}\DecValTok{0}\NormalTok{, scale}\OperatorTok{=}\DecValTok{1}\NormalTok{, moments}\OperatorTok{=}\StringTok{\textquotesingle{}m\textquotesingle{}}\NormalTok{)))}
\end{Highlighting}
\end{Shaded}

\begin{verbatim}
## E(X) = 0.5772156649015329
\end{verbatim}

\begin{Shaded}
\begin{Highlighting}[]
\BuiltInTok{print}\NormalTok{(}\StringTok{"Var(X) = }\SpecialCharTok{\{v\}}\StringTok{"}\NormalTok{.}\BuiltInTok{format}\NormalTok{(v}\OperatorTok{=}\NormalTok{gumbel\_r.stats(loc}\OperatorTok{=}\DecValTok{0}\NormalTok{, scale}\OperatorTok{=}\DecValTok{1}\NormalTok{, moments}\OperatorTok{=}\StringTok{\textquotesingle{}v\textquotesingle{}}\NormalTok{)))}
\end{Highlighting}
\end{Shaded}

\begin{verbatim}
## Var(X) = 1.6449340668482264
\end{verbatim}

Se observa que en este caso la esperanza es la constante de euler
\(\gamma=0.577215664\ldots\).

\end{document}

% Options for packages loaded elsewhere
\PassOptionsToPackage{unicode}{hyperref}
\PassOptionsToPackage{hyphens}{url}
%
\documentclass[
]{article}
\usepackage{amsmath,amssymb}
\usepackage{lmodern}
\usepackage{iftex}
\ifPDFTeX
  \usepackage[T1]{fontenc}
  \usepackage[utf8]{inputenc}
  \usepackage{textcomp} % provide euro and other symbols
\else % if luatex or xetex
  \usepackage{unicode-math}
  \defaultfontfeatures{Scale=MatchLowercase}
  \defaultfontfeatures[\rmfamily]{Ligatures=TeX,Scale=1}
\fi
% Use upquote if available, for straight quotes in verbatim environments
\IfFileExists{upquote.sty}{\usepackage{upquote}}{}
\IfFileExists{microtype.sty}{% use microtype if available
  \usepackage[]{microtype}
  \UseMicrotypeSet[protrusion]{basicmath} % disable protrusion for tt fonts
}{}
\makeatletter
\@ifundefined{KOMAClassName}{% if non-KOMA class
  \IfFileExists{parskip.sty}{%
    \usepackage{parskip}
  }{% else
    \setlength{\parindent}{0pt}
    \setlength{\parskip}{6pt plus 2pt minus 1pt}}
}{% if KOMA class
  \KOMAoptions{parskip=half}}
\makeatother
\usepackage{xcolor}
\usepackage[margin=1in]{geometry}
\usepackage{color}
\usepackage{fancyvrb}
\newcommand{\VerbBar}{|}
\newcommand{\VERB}{\Verb[commandchars=\\\{\}]}
\DefineVerbatimEnvironment{Highlighting}{Verbatim}{commandchars=\\\{\}}
% Add ',fontsize=\small' for more characters per line
\usepackage{framed}
\definecolor{shadecolor}{RGB}{248,248,248}
\newenvironment{Shaded}{\begin{snugshade}}{\end{snugshade}}
\newcommand{\AlertTok}[1]{\textcolor[rgb]{0.94,0.16,0.16}{#1}}
\newcommand{\AnnotationTok}[1]{\textcolor[rgb]{0.56,0.35,0.01}{\textbf{\textit{#1}}}}
\newcommand{\AttributeTok}[1]{\textcolor[rgb]{0.77,0.63,0.00}{#1}}
\newcommand{\BaseNTok}[1]{\textcolor[rgb]{0.00,0.00,0.81}{#1}}
\newcommand{\BuiltInTok}[1]{#1}
\newcommand{\CharTok}[1]{\textcolor[rgb]{0.31,0.60,0.02}{#1}}
\newcommand{\CommentTok}[1]{\textcolor[rgb]{0.56,0.35,0.01}{\textit{#1}}}
\newcommand{\CommentVarTok}[1]{\textcolor[rgb]{0.56,0.35,0.01}{\textbf{\textit{#1}}}}
\newcommand{\ConstantTok}[1]{\textcolor[rgb]{0.00,0.00,0.00}{#1}}
\newcommand{\ControlFlowTok}[1]{\textcolor[rgb]{0.13,0.29,0.53}{\textbf{#1}}}
\newcommand{\DataTypeTok}[1]{\textcolor[rgb]{0.13,0.29,0.53}{#1}}
\newcommand{\DecValTok}[1]{\textcolor[rgb]{0.00,0.00,0.81}{#1}}
\newcommand{\DocumentationTok}[1]{\textcolor[rgb]{0.56,0.35,0.01}{\textbf{\textit{#1}}}}
\newcommand{\ErrorTok}[1]{\textcolor[rgb]{0.64,0.00,0.00}{\textbf{#1}}}
\newcommand{\ExtensionTok}[1]{#1}
\newcommand{\FloatTok}[1]{\textcolor[rgb]{0.00,0.00,0.81}{#1}}
\newcommand{\FunctionTok}[1]{\textcolor[rgb]{0.00,0.00,0.00}{#1}}
\newcommand{\ImportTok}[1]{#1}
\newcommand{\InformationTok}[1]{\textcolor[rgb]{0.56,0.35,0.01}{\textbf{\textit{#1}}}}
\newcommand{\KeywordTok}[1]{\textcolor[rgb]{0.13,0.29,0.53}{\textbf{#1}}}
\newcommand{\NormalTok}[1]{#1}
\newcommand{\OperatorTok}[1]{\textcolor[rgb]{0.81,0.36,0.00}{\textbf{#1}}}
\newcommand{\OtherTok}[1]{\textcolor[rgb]{0.56,0.35,0.01}{#1}}
\newcommand{\PreprocessorTok}[1]{\textcolor[rgb]{0.56,0.35,0.01}{\textit{#1}}}
\newcommand{\RegionMarkerTok}[1]{#1}
\newcommand{\SpecialCharTok}[1]{\textcolor[rgb]{0.00,0.00,0.00}{#1}}
\newcommand{\SpecialStringTok}[1]{\textcolor[rgb]{0.31,0.60,0.02}{#1}}
\newcommand{\StringTok}[1]{\textcolor[rgb]{0.31,0.60,0.02}{#1}}
\newcommand{\VariableTok}[1]{\textcolor[rgb]{0.00,0.00,0.00}{#1}}
\newcommand{\VerbatimStringTok}[1]{\textcolor[rgb]{0.31,0.60,0.02}{#1}}
\newcommand{\WarningTok}[1]{\textcolor[rgb]{0.56,0.35,0.01}{\textbf{\textit{#1}}}}
\usepackage{graphicx}
\makeatletter
\def\maxwidth{\ifdim\Gin@nat@width>\linewidth\linewidth\else\Gin@nat@width\fi}
\def\maxheight{\ifdim\Gin@nat@height>\textheight\textheight\else\Gin@nat@height\fi}
\makeatother
% Scale images if necessary, so that they will not overflow the page
% margins by default, and it is still possible to overwrite the defaults
% using explicit options in \includegraphics[width, height, ...]{}
\setkeys{Gin}{width=\maxwidth,height=\maxheight,keepaspectratio}
% Set default figure placement to htbp
\makeatletter
\def\fps@figure{htbp}
\makeatother
\setlength{\emergencystretch}{3em} % prevent overfull lines
\providecommand{\tightlist}{%
  \setlength{\itemsep}{0pt}\setlength{\parskip}{0pt}}
\setcounter{secnumdepth}{-\maxdimen} % remove section numbering
\usepackage{multirow}
%\PassOptionsToPackage{table,xcdraw}{xcolor}
%\usepackage[usenames,dvipsnames]{xcolor}
\usepackage{colortbl}
\usepackage{makecell}
%\usepackage[spanish,es-nolists]{babel}
%%\usepackage[spanish]{babel}
%% change fontsize of R code
%%colorines
%\usepackage{amsmath,color,array,booktabs,algorithm2e}
%\newcommand\blue[1]{\textcolor{blue}{#1}}
%\newcommand\red[1]{\textcolor{red}{#1}}

%% change fontsize of output
%\let\oldverbatim\verbatim
%\let\endoldverbatim\endverbatim
%\renewenvironment{verbatim}{\scriptsize\oldverbatim}{\endoldverbatim}
%\usepackage{multirow}
%\PassOptionsToPackage{table,xcdraw}{xcolor}
%\usepackage[table,xcdraw]{xcolor}
%\usepackage{colortbl}

\usepackage{makecell}
 \usepackage{booktabs}
 \usepackage{adjustbox}
 \usepackage{amsmath,color,array,booktabs,algorithm2e}
 \definecolor{LightBlue}{RGB}{173,216,230}
 \newcommand\blue[1]{\textcolor{blue}{#1}}
\newcommand\red[1]{\textcolor{red}{#1}}
\newcommand\green[1]{\textcolor{green}{#1}}
 %\setbeamertemplate{navigation symbols}{}
% \setbeamertemplate{footline}[page number]
 \usepackage{mathdots}
\usepackage{yhmath}
\usepackage{mathdots}
\usepackage{MnSymbol,amsmath}
%\renewcommand{\contentsname}{Índice}
%\renewcommand{\chaptername}{Parte}
%\renewcommand{\sectionname}{Lección}
%%%%%%%%%%

%\usepackage[spanish,es-nolists]{babel}
%%\usepackage[spanish]{babel}
%% change fontsize of R code
%%colorines
%\usepackage{amsmath,color,array,booktabs,algorithm2e}
%\newcommand\blue[1]{\textcolor{blue}{#1}}
%\newcommand\red[1]{\textcolor{red}{#1}}

%% change fontsize of output
%\let\oldverbatim\verbatim
%\let\endoldverbatim\endverbatim
%\renewenvironment{verbatim}{\tiny\oldverbatim}{\endoldverbatim}
%\usepackage[table,xcdraw]{xcolor}

\usepackage{booktabs}
\usepackage{longtable}
\usepackage{array}
\usepackage{multirow}
\usepackage{wrapfig}
\usepackage{float}
\usepackage{colortbl}
\usepackage{pdflscape}
\usepackage{tabu}
\usepackage{threeparttable}
\usepackage{threeparttablex}
\usepackage[normalem]{ulem}
\usepackage{makecell}
\usepackage{xcolor}
\ifLuaTeX
  \usepackage{selnolig}  % disable illegal ligatures
\fi
\IfFileExists{bookmark.sty}{\usepackage{bookmark}}{\usepackage{hyperref}}
\IfFileExists{xurl.sty}{\usepackage{xurl}}{} % add URL line breaks if available
\urlstyle{same} % disable monospaced font for URLs
\hypersetup{
  pdftitle={SOLUCIONES: Taller Evaluable 1. Entrega problemas. Probabilidades básico},
  hidelinks,
  pdfcreator={LaTeX via pandoc}}

\title{SOLUCIONES: Taller Evaluable 1. Entrega problemas. Probabilidades
básico}
\author{}
\date{\vspace{-2.5em}}

\begin{document}
\maketitle

\hypertarget{problemas}{%
\section{Problemas}\label{problemas}}

\hypertarget{problema-1}{%
\subsection{\texorpdfstring{Problema 1
(\textbf{1 punto.})}{Problema 1 ()}}\label{problema-1}}

En una urna, hay 5 bolas del mismo tamaño y peso, de los cuales, 3 son
rojas y 2 son azules. ¿De cuántas maneras se pueden extraer los colores
sacando de una en una a una las bolas de la urna?

\hypertarget{soluciuxf3n}{%
\subsubsection{Solución}\label{soluciuxf3n}}

Podemos calcular con permutaciones de dos elementos con repeticiones 3,2
de longitud 5.

\[
PR_5^{3,2}=\binom{5}{3,2}=\frac{5!}{3!\cdot 2!}.
\]

También se puede hacer con combinaciones.

\begin{Shaded}
\begin{Highlighting}[]
\FunctionTok{choose}\NormalTok{(}\DecValTok{5}\NormalTok{,}\DecValTok{3}\NormalTok{)}
\end{Highlighting}
\end{Shaded}

\begin{verbatim}
## [1] 10
\end{verbatim}

\hypertarget{problema-2}{%
\subsection{\texorpdfstring{Problema 2
(\textbf{2 puntos.})}{Problema 2 ()}}\label{problema-2}}

\begin{enumerate}
\def\labelenumi{\alph{enumi}.}
\item
  Cuantos resultados posibles tiene el euromillones: son 5 números
  enteros del 1 al 50 y dos estrellas números enteros del 1 al 12
  \footnote{Sin orden en ambos casos.}.
\item
  Dado un sorteo del euromillones hemos obtenido de números:
  46,47,48,49,50 y de estrellas 1,2. De todos los resultados posibles
  ¿cuántos tienen n=1,2,3,4 aciertos en los números y un acierto en la
  estrella?
\end{enumerate}

\hypertarget{soluciuxf3n-1}{%
\subsubsection{Solución}\label{soluciuxf3n-1}}

\textbf{Apartado a.}

\[
\binom{50}{5}\cdot \binom{12}{2} =
\frac{50!}{(50-5)!\cdot 5!}\cdot 
\frac{12!}{(12-2)!\cdot 2!}=
2118760\cdot 66=
139838160.
\]

\begin{Shaded}
\begin{Highlighting}[]
\FunctionTok{choose}\NormalTok{(}\DecValTok{50}\NormalTok{,}\DecValTok{5}\NormalTok{)}
\end{Highlighting}
\end{Shaded}

\begin{verbatim}
## [1] 2118760
\end{verbatim}

\begin{Shaded}
\begin{Highlighting}[]
\FunctionTok{choose}\NormalTok{(}\DecValTok{12}\NormalTok{,}\DecValTok{2}\NormalTok{)}
\end{Highlighting}
\end{Shaded}

\begin{verbatim}
## [1] 66
\end{verbatim}

\begin{Shaded}
\begin{Highlighting}[]
\FunctionTok{choose}\NormalTok{(}\DecValTok{50}\NormalTok{,}\DecValTok{5}\NormalTok{)}\SpecialCharTok{*}\FunctionTok{choose}\NormalTok{(}\DecValTok{12}\NormalTok{,}\DecValTok{2}\NormalTok{)}
\end{Highlighting}
\end{Shaded}

\begin{verbatim}
## [1] 139838160
\end{verbatim}

\textbf{Apartado b.}

Tenemos \(n=1,2,3,4,5\) aciertos estrelas 50 bolas y 1 acierto en la
estrella

La idea es de los 5 números que han salido elijo \(n\) para acertar son
\(\binom{5}{n}\), para cada elección de \(n\) multiplico por los casos
de los \(5-n\) restantes los tengo que elegir de los que no han salido
que son \(50-5\) que son \(\binom{50-5}{5-n}\). De forma similar la para
las estrellas tengo \(\binom{2}{1}\cdot \binom{11}{1}\) casos. El
producto de estas cantidades me dará en e resultado para \(n=1,2,3,4,5\)
aciertos en las bolas y 1 en las estrellas

\[
\binom{5}{n}\cdot \binom{50-5}{5-n}\cdot \binom{2}{1}\cdot \binom{11}{1}.
\]

Lo podemos calcular manualmente (con calculadora es pesado) o con la
siguiente función de R

\begin{Shaded}
\begin{Highlighting}[]
\NormalTok{loteria}\OtherTok{=}\ControlFlowTok{function}\NormalTok{(n)\{}
\FunctionTok{choose}\NormalTok{(}\DecValTok{5}\NormalTok{,n)}\SpecialCharTok{*}\FunctionTok{choose}\NormalTok{(}\DecValTok{45}\NormalTok{,}\DecValTok{5}\SpecialCharTok{{-}}\NormalTok{n)}\SpecialCharTok{*}\FunctionTok{choose}\NormalTok{(}\DecValTok{2}\NormalTok{,}\DecValTok{1}\NormalTok{)}\SpecialCharTok{*}\FunctionTok{choose}\NormalTok{(}\DecValTok{11}\NormalTok{,}\DecValTok{1}\NormalTok{)}
\NormalTok{\}}
\NormalTok{Casos}\OtherTok{=}\FunctionTok{loteria}\NormalTok{(}\FunctionTok{c}\NormalTok{(}\DecValTok{1}\NormalTok{,}\DecValTok{2}\NormalTok{,}\DecValTok{3}\NormalTok{,}\DecValTok{4}\NormalTok{,}\DecValTok{5}\NormalTok{))}
\NormalTok{Aciertos}\OtherTok{=}\FunctionTok{paste}\NormalTok{(}\StringTok{"Aciertos "}\NormalTok{,}\FunctionTok{c}\NormalTok{(}\DecValTok{1}\NormalTok{,}\DecValTok{2}\NormalTok{,}\DecValTok{3}\NormalTok{,}\DecValTok{4}\NormalTok{,}\DecValTok{5}\NormalTok{),}\StringTok{"bolas y  1 estrella = "}\NormalTok{)}
\NormalTok{casos}\OtherTok{=}\FunctionTok{data.frame}\NormalTok{(Aciertos,Casos)}

\FunctionTok{library}\NormalTok{(kableExtra)}
\NormalTok{knitr}\SpecialCharTok{::}\FunctionTok{kable}\NormalTok{(casos,}\AttributeTok{format=}\StringTok{"latex"}\NormalTok{) }\SpecialCharTok{\%\textgreater{}\%} \FunctionTok{kable\_styling}\NormalTok{(}\AttributeTok{position=}\StringTok{"center"}\NormalTok{, }\AttributeTok{latex\_options =}\NormalTok{ ) }
\end{Highlighting}
\end{Shaded}

\begin{table}
\centering
\begin{tabular}{l|r}
\hline
Aciertos & Casos\\
\hline
Aciertos  1 bolas y  1 estrella = & 16389450\\
\hline
Aciertos  2 bolas y  1 estrella = & 3121800\\
\hline
Aciertos  3 bolas y  1 estrella = & 217800\\
\hline
Aciertos  4 bolas y  1 estrella = & 4950\\
\hline
Aciertos  5 bolas y  1 estrella = & 22\\
\hline
\end{tabular}
\end{table}

\hypertarget{problema-3}{%
\subsection{\texorpdfstring{Problema 3
(\textbf{1 punto.})}{Problema 3 ()}}\label{problema-3}}

Lanzamos un dado de 12 caras numeradas con enteros del 1 al 12 sobre una
mesa plana. Observamos el número superior del dado. Calcular la
probabilidad de que salga mayor que 8 si el resultado es par.

\hypertarget{soluciuxf3n-2}{%
\subsubsection{Solución}\label{soluciuxf3n-2}}

La probabilidad de obtener un valor e de este dado es
\(P(\mbox{Salir} k ,\mbox{en el dado})=\frac{1}{12}.\) para
\(k=1,2,3,\ldots,12.\)

Nos piden \(A_{> 8}=\{9,10,11,12\}\) condicionado a
\(A_{\mbox{par}}={2,4,6,8,10,12}\).

Por un lado tenemos que \(P(A_{> 8})=\frac{4}{12}\) y
\(P(A_{\mbox{par}})=\frac{6}{12}\)

Calculemos lo que se pide

\[
P\left(A_{> 8}| A_{\mbox{par}}\right)=
\frac{ P\left(A_{> 8}\cap A_{\mbox{par}}\right)}{P\left(A_{\mbox{par}}\right)}=
\frac{\frac{4}{12}}{\frac{6}{12}}=\frac{4}{6}=\frac{2}{3}.
\]

\hypertarget{problema-4}{%
\subsection{\texorpdfstring{Problema 4
(\textbf{2 puntos.})}{Problema 4 ()}}\label{problema-4}}

Lanzamos una moneda con probabilidad de cara \(p=\frac{1}{2}\) hasta que
sale cara dos veces o bien la hemos lanzamos 5 veces, lo primero que
ocurra.

Consideremos el experimento con resultados experimentales contar el
número de tiradas

Se pide:

\begin{enumerate}
\def\labelenumi{\alph{enumi}.}
\tightlist
\item
  Describir
  adecuadamente\footnote{Elegir alguna codificación para definir con una lista /tabla el espacio muestral de sucesos $\Omega$.}
  el espacio muestral de todos los sucesos posibles del experimento.
\item
  Si \(A_k\) es el suceso hemos tirado la moneda \(k\) veces calcular
  \(P(A_k)\) para todos casos posibles.
\end{enumerate}

\hypertarget{soluciuxf3n-3}{%
\subsubsection{Solución}\label{soluciuxf3n-3}}

\textbf{Apartado a.} Denotemos por \(C\) el suceso cara y por \(+\) el
suceso cruz. Representemos cada sucesos del experimento por una cadena
ordenada de caras y cruces hasta que se obtengan dos \(C\) o lleguemos a
5 intentos. Los resultados posibles son (16 casos):

\(\Omega=\{ CC, +CC,C+C, ++CC,+C+C,C++C, +++CC,++C+C,+C++C,C+++C, ++++C,+++C+,++C++,+C+++,C++++ +++++\}\)

\textbf{Apartado b.}

\[
\begin{aligned}
P(A_2) & = P(\{CC\}= \left(\frac{1}{2}\right)^2=\frac{1}{4} \\
P(A_3) & = P(\{+CC,C+C\})=\left(\frac{1}{2}\right)^3+\left(\frac{1}{2}\right)^3=2\cdot \frac{1}{8}=\frac{1}{4} \\
P(A_4) & = P(\{++CC,+C+C,C++C,\})=\cdots=3\cdot \left(\frac{1}{2}\right)^4= \frac{3}{16}. \\
P(A_5) = P(
(
\left\{\right.  & 
+++CC,++C+C,+C++C,C+++C,\\
& \left.  ++++C,+++C+,++C++,+C+++,C++++,
+++++
\right\})\\
  & =  10\cdot \left(\frac{1}{2}\right)^5=\frac{10}{32} \\
\end{aligned}
\]

No se pedía pero efectivamente la suma de las probabilidades es 1:

\begin{Shaded}
\begin{Highlighting}[]
\FunctionTok{sum}\NormalTok{(}\FunctionTok{c}\NormalTok{(}\DecValTok{1}\SpecialCharTok{/}\DecValTok{4}\NormalTok{,}\DecValTok{1}\SpecialCharTok{/}\DecValTok{4}\NormalTok{,}\DecValTok{3}\SpecialCharTok{/}\DecValTok{16}\NormalTok{,}\DecValTok{10}\SpecialCharTok{/}\DecValTok{32}\NormalTok{))}
\end{Highlighting}
\end{Shaded}

\begin{verbatim}
## [1] 1
\end{verbatim}

\hypertarget{problema-5}{%
\subsection{\texorpdfstring{Problema 5
(\textbf{2 puntos.})}{Problema 5 ()}}\label{problema-5}}

Tres jugadores \(A,B\) y \(C\) juegan rondas de un juego de cartas. En
cada ronda puede ganar con igual probabilidad cualquiera de los tres
jugadores y el ganador de cada ronda es independientes del anterior.
Gana el juego el primero que consiga ganar tres rondas. Si resulta que
\(A\) gana la primera y la tercera ronda y \(B\) gana la segunda. ¿Cual
es la probabilidad de que \(C\) gane el juego

\hypertarget{soluciones}{%
\subsubsection{Soluciones}\label{soluciones}}

Representemos por \(A\), \(B\) y \(C\) que gana una ronda el jugado
\(A\), \(B\) o \(C\), representemos las partidas ya jugadas en la
siguiente tabla en las siguiente tabla:

\begin{tabular}{|l||l|l|l|}
\hline
Ronda & 1 & 2 & 3\\\hline\hline
Gana A & A & B & A\\ \hline
\end{tabular}

Para que gane \(C\) se tiene que dar alguno de los resultados
siguientes:

\begin{table*}
\centering
\begin{tabular}{|l||l|l|l|l|l|}
\hline
Ronda & 4 & 5 & 6 & 7 & probabilidad \\\hline\hline
Caso 1 Gana C & C & C & C & -  & $\left(\frac{1}{3}\right)^3$\\ \hline
Caso 2 Gana C & B & C & C & C  & $\left(\frac{1}{3}\right)^4$\\ \hline
Caso 3 Gana C & C & B & C & C  & $\left(\frac{1}{3}\right)^4$\\ \hline
Caso 4 Gana C & C & C & B & C  & $\left(\frac{1}{3}\right)^4$\\ \hline
\end{tabular}
\end{table*}

\[P(\mbox{Gana C})= P(\{CCC,BCCC,CBCC,CCBC\})=
\left(\frac{1}{3}\right)^3 + 3\cdot \left(\frac{1}{3}\right)^4=\frac{2}{3}.\]

\hypertarget{problema-6}{%
\subsection{\texorpdfstring{Problema 6
(\textbf{1 punto.})}{Problema 6 ()}}\label{problema-6}}

Sean \(A\) y \(B\) dos sucesos disjuntos e independientes. Demostrar o
justificar si cada una de las siguientes afirmaciones son correctas:

\begin{enumerate}
\def\labelenumi{\alph{enumi}.}
\tightlist
\item
  \(P(B)\neq 0 \Rightarrow P(A|B)> P(B)\).
\item
  \(P(A)\neq 0 \Rightarrow P(B|A)=0\).
\item
  \(P(A\cup B)< P(A)+P(B)\).
\item
  \(P(A)=P(B)=0\).
\end{enumerate}

\hypertarget{soluciuxf3n-4}{%
\subsubsection{Solución}\label{soluciuxf3n-4}}

La afirmación a. es falsa ya que al ser disjuntos \(P(A\cap B)=0\) por
lo que \(P(A|B)=\frac{P(A\cap B)}{P(B)}=0>P(B)\neq 0\) ¡Contradicción!.
La afirmación b. es cierta siempre independientemente del valor de
\(P(A)\) ya que al ser disjuntos \(P(B\cap A)=0\) por lo que
\(P(B|A)=\frac{P(B\cap A)}{P(A)}=0\). La afirmación c.~es falsa ya que
\(P(A\cup B)=P(A)+P(B)-P(A\cap B)=P(A)+P(B)-P(\emptyset)=P(A)+P(B)\)
¡Contradicción! con \(P(A\cup B)< P(A)+P(B)\). La afirmación d.~es falsa
ya que si \(A=\Omega\) suceso seguro y \(B=\emptyset\) suceso imposibles
se cumple que son disjuntos \(A\cap B=\emptyset\) y que
\(P(A\cap B)=P(\emptyset)=P(A)\cdot P(B)=1\cdot 0=0\) !Contraejemplo¡

\hypertarget{problema-7}{%
\subsection{\texorpdfstring{Problema 7
(\textbf{1 punto.})}{Problema 7 ()}}\label{problema-7}}

Sean \(A\), \(B\) y \(C\) tres sucesos tales que \(A\) es independiente
de \(B^c\cap C\) y de \(B\cap C\). Demostrar o justificar si cada una de
las siguientes afirmaciones son correctas:

\begin{enumerate}
\def\labelenumi{\alph{enumi}.}
\tightlist
\item
  \(A,B\) y \(C\) son independientes.
\item
  \(A\) es independiente de \(C\).
\item
  \(A\) es independiente de \(B\)
\end{enumerate}

\hypertarget{soluciuxf3n-5}{%
\subsubsection{Solución}\label{soluciuxf3n-5}}

Aquí partimos de que si la afirmación (a.) es cierta entonces lo son
(b.) y (c.) pues la independencia de una familia de tres sucesos es
equivalente que la probabilidad de la intersección de los tres es igual
al producto de las probabilidades así como todas las intersecciones dos
a dos.

La afirmación (b.) es cierta

\[
\begin{aligned}
P(A\cap B)&=P((A\cap(C\cap B^c))\cup (A\cap (C\cap B))\\
 &=P((A\cap(C\cap B^c))+P(A\cap (C\cap B))-P((A\cap(C\cap B^c))\cap (A\cap (C\cap B))\\
 &=P(a)\cdot P(C\cap B^c)+P(A)\cdot  P(C\cap B))-P(\emptyset)\\
 &=P(A)\cdot (P(C\cap B^c)+P(C\cap B))\\
 &  =  P(A)\cdot P(C).
\end{aligned}
\]

Luego vamos a ver que (c). es falsa así (a) también será falsa:

Vamos a buscar tres sucesos tales que

\(P(A\cap (B^c\cap C)=P(A)\cdot P(B^c\cap C)\) y
\(P(A\cap (B\cap C)=P(A)\cdot P(B\cap C)\).

Consideremos \(\Omega=\{k\in \mathbb{N}| 0< k< 40\}\) un espacio
muestral tal que todos los valores sean equiprobables

Consideremos los siguientes sucesos

\[
\begin{aligned}
A & = \{1,2,3,4,5,6,7,8,9,10,21,22,23,24,25,26,27,28,29,30\}\\
B & = \{6,7,8,9,10,11,12,13,14,15,31,32,33,34,35\},\\
C & = \{1,2,3,4,5,6,7,8,9,10,11,12,13,14,15,16,17,18,19,20\}.
\end{aligned}
\]

Entonces

\[
\begin{aligned}
P(A) & =& \frac{20}{40}=\frac12\\
P(B) & =& \frac{15}{40}=\frac38\\
P(B^c\cap C) & =& \frac{10}{40}=\frac14\\
P(A\cap B) & =& \frac{5}{40}=\frac18\\
P(A\cap B^c\cap C) & =& \frac{5}{40}=\frac18\\
P(A\cap B\cap C) & =& \frac{5}{40}=\frac18
\end{aligned}
\] luego

\[P(A) \cdot P(B^c\cap C)=\frac12\cdot\frac14=\frac18=P(A\cap B^c\cap C).\]
\[P(A) \cdot P(B\cap C)=\frac12\cdot\frac14=\frac18=P(A\cap B\cap C).\]

\ldots{} pero

\[P(A\cap C)=\frac{15}{40}=\frac38\neq P(A)\cdot P(C)=\frac12\cdot \frac12=\frac14.\]

En definitiva hemos obtenido un contraejemplo para la opción (c).

\end{document}
